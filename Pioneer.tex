\chapter{Pioneer Woman}

\begin{minipage}{\linewidth}\rtitle{MOZZARELLA} \index{MOZZARELLA}
	\step{\nicefrac{1}{2} tsp calcium chloride dissolved in 2 Tbsp distilled water
	\\ \nicefrac{1}{2} tablet Rennet Tablet dissolved in \nicefrac{1}{4} cup distilled water
	\\ 1 gallon whole milk
	\\ 2 tsp citric acid
	\\ \nicefrac{1}{2} tsp flaked salt}{Heat milk slowly over medium heat. Sprinkle in citric acid and dissolved calcium chloride. Heat until milk reaches 88${}^\circ$F. Curd should be forming. Stir in dissolved rennet. Heat slowly to 105${}^\circ$F. Remove from heat and let sit covered for 20 minutes. Curd and whey should be fully separated. Transfer curd to microwave safe dish. Press curds to squeeze out as much whey as possible. Microwave on high for one minute. Press curds to remove whey. Repeat for 20--30 seconds until cheese begins to mass together and becomes sticky. Add salt a little and knead until cheese becomes smooth and shiny. Microwave on high for one more minute. Drain. Knead. When cheese stretches like taffy, pull and stretch until it is completely cooled.}

\end{minipage}\par\begin{minipage}{\linewidth}\rtitle{\hypertarget{fetalink}{FETA}} \index{FETA}
\step{3 gallons milk
\\ \nicefrac{1}{4} tsp Mesophilic culture
\\ \nicefrac{1}{4} tsp lipase powder
\\ 1 tsp single-strength liquid rennet dissolved in \nicefrac{1}{2} cup of distilled water
\\ kosher salt (no substitute)
\stepspace  BRINE:
\\ \nicefrac{1}{2} cup kosher salt (no substitute)
\\ 1 gallon distilled water}{Bring milk to 86--88$^\circ$F. Add mesophilic culture and lipase powder. Stir well. Cover pot and let rest for one hour at 86--88$^\circ$F. Add dissolved rennet and stir vigorously for 15--20 seconds. Cover pot and let it stand undisturbed for 30-40 minutes or until the curd 'breaks' cleanly. Cut a \nicefrac{1}{2}-inch grid pattern into the curd. Let the curd rest undisturbed for 10 minutes. After 10 minutes, stir gently. Keep the curd at 86--88$^\circ$F for 45 minutes, stirring occasionally. Drain curds in cheesecloth for 3--4 hours. Flip cheese and rehang. Drain for 24 hours. Cut it into blocks. Generously sprinkle all surfaces of cheese with kosher salt, place in sterilized container with a lid.
Let the cheese rest at room temperature for 2 to 3 days. Let brine cool before pouring it over the cheese. Store cheese in the refrigerator. Let age at least two weeks.}

%\end{minipage}\par\begin{minipage}{\linewidth}\rtitle{FETA 2} \index{FETA 2}
	%\step{2 Gallons of Whole Milk
%\\ \nicefrac{1}{4} tsp MM 100
%\\ \nicefrac{1}{4} tsp lipase powder
%\\ 1 tsp liquid rennet
%\\ \nicefrac{1}{4} cup distilled water
%\\ Kosher salt}{Make the brine: \nicefrac{1}{2} cup kosher salt per \nicefrac{1}{2} gallon of water (boiled and cooled to room temperature). Dissolve rennet in distilled water. Warm milk to 88${}^\circ$F (86${}^\circ$F for goat milk). Remove from heat. Add culture and lipase. Stir well and let ripen, covered, for 1 hour. Add rennet and stir briskly for 15 seconds. Cover and let set 30 to 40 minutes, or until you see a clean break. Cut \nicefrac{1}{2}-inch checkerboard pattern. Hold the knife at a 45${}^\circ$ angle and retrace your cuts. Let the curds rest 5 minutes (10 minutes). Stir curd gently and cut any pieces you missed. Hold milk at 88${}^\circ$F (86${}^\circ$F) for 45 minutes; keeping the pot covered will maintain the temperature. If the temperature falls, place the pot in a sink of 86-degree water to raise the temperature. Stir occasionally. Remove curd into cheesecloth and hang to drain for 3-4 hours. Flip it and continue draining for 24 hours. Cut it into usable, 2-inch cubes. Sprinkle all sides of the curds with kosher salt and place them in a sterilized, sealable container. Cover and let sit at room temperature for 2 to 3 days to harden up the blocks. The blocks will continue to release whey during this time. Transfer the blocks to a large sterilized glass container, such as a glass pickle jar, and add the brine. Age the cheese in brine for 1 to 4 weeks before use.}

\end{minipage}\par\begin{minipage}{\linewidth}\rtitle{HALLOUMI} \index{HALLOUMI}
\step{1 gallon of milk
\\ \nicefrac{1}{4} tsp MA4002
\\ 3/8 tsp Calcium Chloride
\\ \nicefrac{1}{4} tsp Single Strength Liquid Rennet dissolved in \nicefrac{1}{4} cup distilled water
\\ \nicefrac{1}{2} oz Salt
\\ \nicefrac{1}{8} tsp Citric Acid
\\ 1 pint of milk}{Heat milk to 86--88${}^\circ$F. Add culture. Add lipase and/or calcium chloride. Stir briefly to incorporate well into milk. Add rennet. Wait 30--40 minutes for milk to separate (should start at 15--20 minutes). Cut curd with a knife to see if it splits naturally. If it's ready, cut curd vertically into 0.75--1.5 inch squares. Rest for 3-5 minutes. Cut horizontally with spoon. Stir gently, increasing heat to 100--106${}^\circ$F for 20-30 minutes. Keep temperature steady for 20-30 minutes with intermittent stirring every 3-5 minutes. Allow curds to settle for 5 minutes. Remove curds from whey. Heat whey to 185-195${}^\circ$F (do not let it boil). Put curds into form and presss with 1-2 pound weight. Turn at 15--20 minute intervals. When whey reaches 150${}^\circ$F, add citric acid for ricotta. At 165-170${}^\circ$F, add 1 tsp of salt and a pint of milk. As whey reaches 185-195${}^\circ$F, stop stirring and allow ricotta to rise to the top for about 10 minutes. Skim off curds. Drain. Heat whey to 190--195${}^\circ$F. Lower halloumi into whey and heat for 30-40 minutes. When it floats, it is ready. Drain and cool. As it cools, you can flatten and sprinkle it with salt.}

\end{minipage}\par\begin{minipage}{\linewidth}\rtitle{\hypertarget{paneerlink}{PANEER}} \index{PANEER}
\step{1 gallon whole milk
	\\ 3 Tbsp lemon juice}{Bring milk to boil. As soon as it's just about to boil, add lemon juice and reduce heat. Collect curds in a cheese cloth, trying not to break them up. Rinse curds to remove lemon flavour. Hang to let whey drain out for about 1 hour. Put cheese in cloth on a plate with a weight on it for about 2 hours. Refrigerate. Cut into shape. Store in freezer.}

\end{minipage}\par\begin{minipage}{\linewidth}\rtitle{BILTONG} \index{BILTONG}
\name{(A South African jerky that we grew to love while in Kenya)}\\
\step{2 kg beef (rump, topside or silverside)
\\ \nicefrac{1}{2} cup whole coriander seeds
\\ 300 mL apple cider vinegar
\\ coarse salt
\\ 2 Tbsp black peppercorns
\\ 3 Tbsp ground coriander
\\ 1 Tbsp ground black pepper}{Slice beef into steaks an inch thick and trim fat. Rub coarse salt into each side of steaks and drain on a wire rack. Leave for 1~\nicefrac{1}{2} hours. Toast coriander. Roughly grind all whole spices and add remaining spices. Scrape steaks with a knife to remove salt. Marinate each piece of beef in the vinegar for 2--4 minutes. If the beef is not fully covered then turn it over half way through. Cover steaks with seasoning. Leave to marinate for 5--6 hours. Dehydrate at 65 degrees for 12 hours. Check after 7--8 hours and adjust time accordingly.}

\end{minipage}\par\begin{minipage}{\linewidth}\rtitle{PICKLED BEETS} \index{PICKLED BEETS} \index{pickles}
Cook small beets or cook and slice larger beets into a 2 quart sealer. Cover with 1 cup vinegar, \nicefrac{3}{4} cup sugar, 1 cup water and seal.

\end{minipage}\par\begin{minipage}{\linewidth}\rtitle{BREAD AND BUTTER PICKLES} \index{BREAD AND BUTTER PICKLES} \index{pickles}
\step{30 medium sized cucumbers	
	\\ 5 cups sugar
	\\ 8 large white onions	
	\\ 2 Tbsp mustard seed
	\\ 2 sweet peppers	
	\\ 1 tsp turmeric
	\\ \nicefrac{1}{2} cup salt	
	\\ \nicefrac{1}{2} tsp cloves
	\\ 5 cups vinegar
		}{
Wash cucumbers and slice as thin as possible. Chop onions and peppers. Combine with cucumbers and salt and let stand for 3 hours. Drain. Combine vinegar, sugar and spices in large kettle. Bring to a boil. Add drained cucumbers. Heat thoroughly but do not boil. Pack hot in hot jars and seal.}

\end{minipage}\par\begin{minipage}{\linewidth}\rtitle{CHOCHO PICKLES}  \index{CHOCHO PICKLES} \index{pickles}
\name{(Jungle Camp Cookbook)} \\
\step{6-8 cups raw chayotes
	\\ 1 medium onion
	\\ 2 Tbsp coarse salt
	\\ 1 \nicefrac{1}{4} cup mild vinegar	
	\\ 1 \nicefrac{1}{4} cup sugar
	\\ \nicefrac{3}{8} tsp turmeric	
	\\ \nicefrac{3}{4} tsp prepared mustard or 1 \nicefrac{1}{2} tsp mustard seed
	\\ \nicefrac{1}{8} tsp ground cloves}{
Slice chayotes and onions. Sprinkle salt over the vegetables and mix well. Let stand for 2 hours. In large saucepan bring remaining ingredients to a boil. When syrup is boiling, add the vegetables gradually with very little stirring. Return to boiling point and cook for 1 minutes. Put into sterilized jars.}

\end{minipage}\par\begin{minipage}{\linewidth}\rtitle{DILL PICKLES}  \index{DILL PICKLES} \index{pickles}
\name{(Loraine Hiebert)}\\
\step{24 3-4" cucumbers
\\ dill
\\ \nicefrac{1}{2} red pepper
\\ 2 cloves garlic
\stepspace BRINE:
\\ 1 cup vinegar
\\ 2 cups water
\\ \nicefrac{1}{4} cup pickling salt
\\ 1 Tbsp sugar}{Wash and cut off the flower end of cucumbers. Cover with ice water and let stand for 2 hours to make the cukes firm. To each jar add dill, red pepper, and garlic. Bring brine to a boil. Pack whole or sliced cucumbers into jars. Fill the jars with the hot brine. Seal. Makes 2 quarts of pickles.}

\end{minipage}\par\begin{minipage}{\linewidth}\rtitle{SALSA}  \index{SALSA}
\step{1 red onion, chopped
\\ 1 white onion, chopped
\\ 1 yellow onion, chopped
\\ 6 pounds fresh tomatoes, peeled and chopped
\\ 2 banana peppers, chopped*
\\ 3 green bell peppers, chopped*
\\ 3 6-oz cans tomato paste
\\ \nicefrac{1}{2} cup white vinegar
\\ 5-6 cloves garlic, minced
\\ 2 Tbsp fresh cilantro
\\ 1 \nicefrac{1}{2} Tbsp salt
\\ \nicefrac{1}{2} Tbsp cayenne pepper
\\ 2 tsp ground cumin
\\ 2 tsp New Mexico green chili
\\ 3 Tbsp lime juice
\\ \nicefrac{1}{4} cup brown sugar
\\ \nicefrac{1}{2} cup white sugar}{Combine all ingredients.  Allow the ingredients to drain in a colander for at least 1 hour.  Boil until you have the right consistency (\nicefrac{1}{2} hour -- 3 hours).  Pour into hot sterilized jars and seal.
*The total amount of peppers should be around 2 \nicefrac{1}{2} cups.}

\end{minipage}\par\begin{minipage}{\linewidth}\rtitle{RED PEPPER JELLY} \index{RED PEPPER JELLY}
\name{(Kelly Penner)}\\
\step{1 large red bell pepper \\
1 large green bell pepper \\
10 jalapeno peppers \\
1 \nicefrac{1}{2} cups white vinegar \\
\nicefrac{1}{2} tsp salt \\
6 cups white sugar \\
1 pouch certo liquid fruit pectin}{
Chop peppers and put in large pot with vinegar, salt and sugar. Boil 10 minutes. Add pectin and boil 1 minutes longer. Fill canny jars and seal. Great on crackers with cream cheese or as a dip for tortilla chips or chicken fingers.}


\end{minipage}\par\begin{minipage}{\linewidth}\rtitle{GRANOLA (GF*)} \index{GRANOLA (GF*)} \index{Gluten Free!granola}
\step{3 \nicefrac{1}{2} cups rolled oats
\\ \nicefrac{1}{2} cup sorghum flour
\\ \nicefrac{1}{2} cup GF oat flour
\\ 2 Tbsp ground flax seed
\\ 2 Tbsp sesame seeds
\\ \nicefrac{1}{2} cup coconut chips (large flakes)
\\ \nicefrac{1}{2} tsp cinnamon
\\ \nicefrac{1}{4}-\nicefrac{1}{2} cup brown sugar, packed
\\ \nicefrac{1}{4} tsp salt
\\ \nicefrac{3}{4} cup unsalted nuts (such as sliced almonds or chopped pecans)
\\ \nicefrac{1}{4} cup raw sunflower seeds
\\ \nicefrac{1}{4} cup oil
\\ \nicefrac{1}{4} cup honey
\\ 1-2 Tbsp peanut butter
\\ 2 Tbsp vanilla
\\ \nicefrac{1}{2} cup warm water
\\ 1 cup dried fruit}{Preheat oven to 275${}^\circ$F. Line a large baking sheet with parchment paper. In a large bowl, stir together all the dry ingredients (minus the fruit). In a large measuring cup, whisk together all wet ingredients. Pour the wet ingredients over the dry and stir until evenly moistened. Spread in an even layer on the baking sheet. Bake for 15 minutes. stir. Return granola to oven and continue to bake, stirring every 10 minutes, until the granola is golden and dry, about 50 minutes. total baking time. Remove the granola from the oven and immediately sprinkle with the dried fruit. Stir. Allow to cool completely before storing in an air-tight container.}

\end{minipage}\par\begin{minipage}{\linewidth}\rtitle{GF JULES FLOUR} \index{GF JULES FLOUR} \index{Gluten Free!flours}
\step{4 tsp xanthan gum
\\ \nicefrac{1}{2} cup cornstarch
\\ 1 cup tapioca starch
\\ 1 cup potato starch
\\ 1 cup white rice flour
\\ \nicefrac{1}{2} cup cornflour}{}

\end{minipage}\par\begin{minipage}{\linewidth}\rtitle{CUP 4 CUP GF FLOUR} \index{CUP 4 CUP GF FLOUR} \index{Gluten Free!flours}
\step{87 g white rice flour (31\%) \\
70 g cornstarch (25\%) \\
42 g tapioca starch (15\%) \\
39 g brown rice flour (14\%) \\
28 g milk powder (10\%) \\
8.5 g potato starch (3\%) \\
5.6 g xanthan gum (2\%) \\
}{Yield: 2 cups.}

\end{minipage}