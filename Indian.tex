\chapter{Indian}

\begin{minipage}{\linewidth}\rtitle{SHAMBANI CURRY} \index{SHAMBANI CURRY}
My own recipe adapted from the Shambani Cottages cooks at Mombasa \\
	\step{1 lb chicken, beef, or pork, cubed
	\\ 2--3 kg tomatoes, peeled, seeded, and diced
	\\ 1 large onion, diced
	\\ 2 cloves garlic, crushed
	\\ 2 bunches cilantro (cilantro)
	\\ 3 Tbsp curry powder
	\\ \nicefrac{1}{2} tsp cumin, turmeric, cardamom, cloves, ground coriander
	\\ 1--2 bouillon cubes}{
Stir fry meat in a little oil. Remove from pan and set aside. Add tomatoes, onion, garlic, and cilantro to the drippings. 
Saut\'e. If necessary, add a little oil. Add spices and bouillon cube. Let the vegetables and spices simmer until sauce thickens (approx 30 minutes. Longer if the tomatoes were juicy). Add the cooked meat. This sauce is also good with shrimp or vegetables. Serve over rice with yoghurt, diced banana, mango, papaya, coconut, peanuts, chopped cooked egg, etc.}

\end{minipage}\par\begin{minipage}{\linewidth}\rtitle{CHICKEN CURRY} \index{CHICKEN CURRY}
 \textit{(Diane DeVries)}\\
	\step{2 Tbsp butter or margarine	
	\\ 1 \nicefrac{1}{2} cups finely chopped apple
	\\ \nicefrac{1}{2} cup chopped onion	
	\\ 1 clove garlic, minced
	\\ 2 Tbsp flour	
	\\ 2 Tbsp curry powder
	\\ 1 tsp salt	cilantro
	\\ \nicefrac{1}{2} tsp cumin	
	\\ \nicefrac{1}{4} tsp nutmeg
	\\ 1 chicken bouillon cube	
	\\ 2 cups milk
	\\ 2 cups cooked, cubed chicken}{
Melt butter. Add apples, onions and garlic. Cook 5 minutes. Stir in flour, spices and salt. Slowly blend in milk. Cook and stir until thick. Add chicken. Heat through.}

\end{minipage}\par\begin{minipage}{\linewidth}\rtitle{INDIAN CHICKEN} \index{INDIAN CHICKEN}
 \textit{(More with Less Cookbook)\\
Not authentic, but really delicious.}\\
	\step{2 Tbsp margarine
	\\ 2 Tbsp vegetable oil
	\\ flour
	\\ 1 3-lb chicken, cut up
	\\ 2 medium onions, chopped
	\\ 3 Tbsp flour	
	\\ 2 Tbsp curry
	\\ 2 tsp salt	
	\\ 1 tsp ginger
	\\ \nicefrac{1}{3} cup honey	
	\\ \nicefrac{1}{4} cup soy sauce
	\\ 3 cups chicken broth or bouillon}{
Heat margarine and oil in large skillet. Flour and fry chicken until browned. Remove from skillet and place in casserole.
In remaining fat, saut\'e onions. Add flour, curry, salt, and ginger. Combine honey, soy sauce, and chicken broth and then add to onion mixture. Cook over high heat until sauce thickens. Pour sauce over chicken and bake, covered, for 1 hour at 350$^\circ$F.}

\end{minipage}\par\begin{minipage}{\linewidth}\rtitle{CHICKEN TIKKA MASALA} \index{CHICKEN TIKKA MASALA}
\step{2 Tbsp ghee
\\ 1 onion, finely chopped
\\ 4 cloves garlic, minced
\\ 1 Tbsp ground cumin
\\ 1 tsp salt
\\ 1 tsp ground ginger
\\ 1 tsp cayenne pepper
\\ \nicefrac{1}{2} tsp cinnamon
\\ \nicefrac{1}{4} tsp turmeric
\\ 1 -- 14 oz can tomato sauce
\\ 1 cup heavy whipping cream
\\ 2 tsp paprika
\\ 1 Tbsp sugar
\\ 1 Tbsp veg oil
\\ 4 chicken breast halves, cut into bite size pieces
\\ \nicefrac{1}{2} tsp curry powder
\\ \nicefrac{1}{2} tsp salt (opt)
\\ 1 tsp sugar (opt)}{Heat ghee in skillet over med heat.  Cook onion until translucent. Stir in garlic. Cook until fragrant. Add cumin, salt, cayenne pepper, ginger, cinnamon and turmeric. Fry about 2 minutes. Stir in tomato sauce. Bring to boil and simmer 10 minutes. Add cream, paprika and 1 tbsp sugar. Simmer until sauce is thickened, about 10--15 minutes.
Heat veg oil in separate skillet. Stir chicken into hot oil. Sprinkle with curry powder. Fry chicken until lightly browned but still pink inside, about 3 minutes. Transfer chicken and pan juices into the sauce. Simmer about 30 minutes. Adjust salt and sugar to taste.}


\end{minipage}\par\begin{minipage}{\linewidth}\rtitle{TANDOORI CHICKEN} \index{TANDOORI CHICKEN}
\step{\nicefrac{3}{4} cup finely chopped onion	
	\\ 1 tsp finely chopped peeled fresh ginger
	\\ 2 garlic cloves, peeled	
	\\ \nicefrac{1}{2} cup plain yoghurt
	\\ 1 Tbsp fresh lemon juice	
	\\ 1 tsp paprika
	\\ 1 tsp ground cumin	
	\\ 1 tsp ground coriander seeds
	\\ \nicefrac{1}{2} tsp salt	
	\\ \nicefrac{1}{2} tsp chilli powder
	\\ \nicefrac{1}{4} tsp black pepper	
	\\ dash of ground nutmeg
	\\ 4 chicken breast halves}{
Mix marinade ingredients. Make 3 diagonal cuts \nicefrac{1}{4}-inch deep across top of each chicken breast half. Put chicken in marinade. Cover and refrigerate 8 hours or overnight. Grill chicken 6 minutes on each side or until chicken is done.}

\end{minipage}\par\begin{minipage}{\linewidth}\rtitle{PALAK CHICKEN} \index{PALAK CHICKEN}
	\step{1 large chicken	
		\\ 3 Tbsp ghee or clarified butter
	\\ 2 onions, sliced	
	\\ 4-5 cloves garlic, chopped
	\\ 4 cm fresh ginger, chopped	
	\\ 3 cm stick cinnamon
	\\ 4-5 cardamoms	
	\\ 2 tsp coriander powder
	\\ \nicefrac{1}{2} kg fresh spinach	
	\\ 4-5 chopped tomatoes, peeled
	\\ 2-3 green chillies}{
In a large frying pan heat the ghee or clarified butter. Add the onions and saut\'e until golden brown. Grind or blend together the spices and stir into the onions, coating them with the spices. Add the chicken and lightly brown. Add chopped spinach, peeled tomatoes and chillies. Cover and cook over low heat about 1 hour until the meat is tender. Serves 4.}

\end{minipage}\par\begin{minipage}{\linewidth}\rtitle{CHICKEN MAKHANI} \index{CHICKEN MAKHANI}
	\step{2 Tbsp lemon juice	
		\\ 1 pound chicken, cut into bite-size pieces
		\\ 1 tsp garam masala	
		\\ \nicefrac{1}{4} tsp fenugreek
		\\ 1 pinch cayenne pepper	
		\\ salt
		\\ 1 Tbsp oil	
		\\ \nicefrac{1}{2} onion, finely chopped
		\\ 2 Tbsp butter	
		\\ 1 Tbsp lemon juice
		\\ \nicefrac{1}{2} Tbsp ginger, grated	
		\\ \nicefrac{1}{2} Tbsp garlic, crushed
		\\ 1 tsp garam masala	
		\\ 1 tsp chili powder
		\\ 1 tsp ground cumin	
		\\ \nicefrac{1}{4} tsp fenugreek
		\\ 1 bay leaf	
		\\ \nicefrac{1}{4} cup plain yoghurt
		\\ 1 cup heavy cream	
		\\ 1 cup tomato puree
		\\ \nicefrac{1}{3} cup cashews, pureed	
		\\ salt}{
Combine first 6 ingredients and let chicken marinate for a minimum of 30 minutes.
Heat oil in large saucepan over medium high heat. Saut\'e onion in medium high heat until soft and translucent. Stir in butter, lemon juice, ginger, garlic, and spices. Stir for 1 minute. Add tomato puree and cook for 2 minutes, stirring frequently. Stir in cream, yoghurt, and cashews. Reduce heat to low and simmer, stirring frequently.
Heat 1 Tbsp oil in heavy skillet over medium heat. Cook chicken. Add chicken to sauce. Let simmer.}

\end{minipage}\par\begin{minipage}{\linewidth}\rtitle{MURG MAKHANI} \index{MURG MAKHANI}
\textit{(Neha Gupta, Katrina's cooking teacher in Delhi)}\\
\step{1 Tbsp ghee \\
\nicefrac{1}{2} tsp garlic, minced \\
\nicefrac{1}{2} tsp ginger, minced \\
250g chicken, cubed \\
1 large red onion, minced \\
4 tomatoes, pur\'eed \\
1 tsp salt \\
\nicefrac{1}{2} tsp coriander \\
\nicefrac{1}{2} tsp red chile \\
\nicefrac{1}{2} tsp turmeric \\
\nicefrac{1}{2} tsp garam masala \\
\nicefrac{1}{2} tsp tandoori masala \\
1 tsp dried fenugreek leaves \\
1 cup whole milk \\
2 tsp cream \\
\nicefrac{1}{2} tsp garam masala \\
\nicefrac{1}{2} tsp tandoori masala}{Saut\'e garlic and ginger in ghee. Add onions and fry till translucent. Add tomato and cook until boiling. Add all spices and salt, except \nicefrac{1}{2} tsp garam masala and \nicefrac{1}{2} tsp tandoori masala. Add fenugreek, milk, and chicken and cook covered for 5 minutes, stirring occasionally. Add remaining spices and cream. Optional: Separately brown chicken with spices before adding to gravy.}

\end{minipage}\par\begin{minipage}{\linewidth}\rtitle{MADRAS FISH CURRY} \index{MADRAS FISH CURRY}
	\step{\nicefrac{1}{2} kg fish fillets	
		\\ \nicefrac{1}{2} tsp cumin seeds
	\\ \nicefrac{1}{2} tsp mustard seeds	
	\\ 1 green chilli
	\\ \nicefrac{1}{2} tsp paprika	
	\\ \nicefrac{1}{2} tsp turmeric powder
	\\ 1 onion sliced	
	\\ \nicefrac{1}{2} cup water
	\\ juice of 2 lemons	
	\\ 2 Tbsp ghee for frying
	\\ salt to taste}{In a flat dish, place the fish fillets. Mix together the marinade ingredients and pour over the fish. Marinate at least 2-3 hours. Grind or blend together spices. Heat ghee or oil in frying pan and saut\'e fish, onions and spices until the onions and fish are coated with spices. Add the lemon juice, water and salt and cook until tender.}
	
	\step{Marinade:
	\\ 2 Tbsp vinegar	
	\\ 1 Tbsp prepared chilli sauce
	\\ 1 tsp salt}{}

\end{minipage}\par\begin{minipage}{\linewidth}\rtitle{MAHI KASHMIRI} \index{MAHI KASHMIRI}
	\step{500 grams fish	
	\\ 2 cups yoghurt
\\ 2 tsp grated ginger	
\\ 1 tsp cumin seeds
\\ 2 tsp coriander powder	
\\ 4 green chilies, seeded and slit
\\ 2 cloves	
\\ 1 tsp garam masala
\\ 2 Tbsp fresh coriander	
\\ 3 Tbsp ghee
\\ 1 cup water	
\\ salt to taste
\\ Pinch of asafetida	
\\ Pinch of turmeric}{
Rub fish slices with salt and turmeric. Set aside for 5 minutes. Whisk yoghurt with a little salt. Heat ghee. Fry fish till golden. Heat remaining ghee. Add all spices except chopped coriander. Add yoghurt. Cook till reddish brown. Add fish and water. Simmer 20 minutes. Add green chilies and fresh coriander. Simmer, covered, 10 minutes.}

\end{minipage}\par\begin{minipage}{\linewidth}\rtitle{PORK VINDALOO} \index{PORK VINDALOO}
\textit{(Western India)} \\
\step{2 Tbsp coriander seed	
	\\ 1 tsp cumin seeds
	\\ 2 cardamoms	
	\\ 3 cm piece cinnamon
	\\ 6 cloves	
	\\ 6 peppercorns
	\\ 2 tsp turmeric	
	\\ 1 tsp chopped fresh ginger
	\\ 1-2 tsp chilli powder	
	\\ 1 tsp salt
	\\ 1 kg pork cubes	
	\\ 2 \nicefrac{1}{2} cups vinegar
	\\ 2 bay leaves	
	\\ 1 Tbsp ghee or oil
	\\ 6 cloves garlic, chopped	
	\\ 1 Tbsp ghee or oil
	\\ 2 tsp mustard seeds}{In hot frying pan, lightly brown coriander and cumin seeds. Grind or crush together with the rest of the spices to form a paste, to which you add a tsp of vinegar. Put aside. Dilute \nicefrac{1}{2} cup of vinegar with \nicefrac{1}{2} cup of water and wash the pork in the mixture. Drain and coat the meat with the spice paste. Sprinkle with broken bay leaves and pour the 2 cups of vinegar over the meat and allow to marinate overnight. The next day, fry the chopped garlic in ghee for a minute and add the mustard seeds. Fry for a few minutes. Add the meat and marinade and simmer gently about 1 \nicefrac{1}{2} hours until tender. This dish can be served hot or cold.}

\end{minipage}\par\begin{minipage}{\linewidth}\rtitle{KEBAB CURRY} \index{KEBAB CURRY}
	\step{1 lb beef, mutton or veal, cubed	
		\\ 3 inches fresh ginger
	\\ 4 onions, quartered	
	\\ 2-3 cloves garlic, chopped
	\\ 1 Tbsp curry powder	
	\\ \nicefrac{1}{2} tsp garam masala
	\\ \nicefrac{1}{2} tsp cumin	
	\\ 3-5 green chillies
	\\ 1 cup beef stock or hot water	
	\\ 5-6 tomatoes, peeled and quartered
	\\ handful fresh coriander leaves}{
Peel and slice ginger into pieces which can be threaded on to a skewer. Thread the meat, quartered onions and ginger on 4 skewers. In a frying pan, heat the butter and stir in any remaining onions and chopped garlic. Add spices and finely chopped chillies. Fry for 3 minutes. Add the skewers, the stock and tomatoes. Simmer about 1 hour until the meat is tender. Serve the skewers sprinkles with coriander leaves.}

\end{minipage}\par\begin{minipage}{\linewidth}\rtitle{DAHL} \index{DAHL}
	\step{1 cup red lentils	
		\\ 2 cups water
	\\ 1 cup chopped onion	
	\\ 1 large tomato, diced
	\\ 2 garlic cloves, minced	
	\\ \nicefrac{1}{4} tsp turmeric
	\\ \nicefrac{1}{2}--1 tsp cayenne pepper	
	\\ \nicefrac{3}{4} tsp salt
	\\ 1 tsp ground coriander}{
Combine ingredients in saucepan. Cover and cook for about 20 minutes until vegetables are tender. Run through blender or serve as is.}

\end{minipage}\par\begin{minipage}{\linewidth}\rtitle{MUSHROOM DAHL} \index{MUSHROOM DAHL}
\textit{(Leslie Sherman)} \\
	\step{2 cups red lentils	
		\\ \nicefrac{1}{2} pound mushrooms
	\\ \nicefrac{1}{2} tsp turmeric	
	\\ 4 Tbsp tamarind paste
	\\ 2 Tbsp cilantro	
	\\ \nicefrac{1}{2} tsp sugar
	\\ 2 tsp salt	
	\\ 3 Tbsp oil
	\\ 10 whole black peppercorns	
	\\ \nicefrac{1}{2} tsp whole black mustard seeds
	\\ 10 whole fenugreek seeds	
	\\ 1--3 whole dried hot red peppers}{
Rinse lentils. Put in a pot with water and bring to boil. Remove scum. Simmer. When nearing completion, add turmeric, cilantro, mushrooms, salt, sugar, and tamarind paste. In a separate skillet, heat oil over high heat, add peppercorns, mustard, fenugreek, and peppers. When mustard seeds begin to pop, add to the pot with lentils.}

\end{minipage}\par\begin{minipage}{\linewidth}\rtitle{CHANA MASALA} \index{CHANA MASALA}
\textit{(Neha Gupta, Katrina's cooking teacher in Delhi)}\\
\step{2 Tbsp oil
\\ 1 black teabag
\\ 100 g chickpeas
\\ \nicefrac{1}{2} tsp garlic, minced
\\ 1 medium onion, finely chopped
\\ 1 tomato, finely chopped
\\ 3 Tbsp tamarind paste
\\ 3 Tbsp tomato pur\'ee
\\ 2 green cardamom pods, crushed
\\ 2 black cardamom pods, crushed
\\ 1 dried red pepper
\\ 2 bay leaves
\\ 1 tsp turmeric
\\ 1 tsp garam masala
\\ 1 tsp red chile
\\ 1 tsp chaat masala
\\ 1 tsp salt}{Soak chickpeas in water and leave overnight. Pressure cook chickpeas with a tea bag. Sauce whole spices in oil. Add garlic and onion and saut\'e until translucent. Add tomato. Add tamarind paste and cook 3--4 minutes. Add tomato pur\'ee and cook 3--4 minutes. Add chickpeas and mash a bit. Add remaining spices and salt. Cook covered for 5 minutes, stirring occasionally.}

\end{minipage}\par\begin{minipage}{\linewidth}\rtitle{AVIAL} \index{AVIAL}
\step{4 cups vegetables such as green beans, potatoes, carrots, kohlrabi, green plantain, cubed
\\ 2 cups water	
\\ 1 tsp salt
\\ 2 Tbsp water	
\\ 114 ml canned chopped green chillies, drained
\\ 1 Tbsp chopped fresh coriander	
\\ 1 tsp ginger paste
\\ \nicefrac{1}{2} cup coconut powder or fine coconut	
\\ 1 tsp cumin seeds
\\ \nicefrac{1}{4} tsp turmeric	
\\ \nicefrac{1}{8}-\nicefrac{1}{4} tsp cayenne pepper
\\ \nicefrac{1}{3} cup plain yoghurt}{Cook vegetables in water and salt for about 10 minutes, until tender. Drain. Grind next 8 ingredients into a paste in blender. Add to vegetables. Cool. Stir in yoghurt. Serve with roti or chapati or rice.}

\end{minipage}\par\begin{minipage}{\linewidth}\rtitle{GOBI SABJI} \index{GOBI SABJI}
	\step{2 Tbsp oil or clarified butter
\\ 3 tomatoes, peeled and chopped
\\ 1 onions, finely chopped
\\ 1 jalapeno
\\ 1 Tbsp ginger
\\ 2 cloves garlic, minced
\\ \nicefrac{1}{2} head cauliflower, cut into florets
\\ 1 Tbsp turmeric powder
\\ 1 tsp salt}{Heat oil in frying pan. Add onions and cook until soft. Add tomatoes and cook until it forms a good masala. Add jalapeno, ginger, and garlic and cook until fragrant. Add cauliflower and cover. Add turmeric and salt when cauliflower is partially cooked. Add water as necessary until cauliflower turns soft.}

\end{minipage}\par\begin{minipage}{\linewidth}\rtitle{GOBI ALOO} \index{GOBI ALOO}
	\step{2 Tbsp oil or clarified butter
\\ \nicefrac{1}{2} tsp whole mustard seeds
\\ 3 tomatoes, peeled and chopped
\\ 2 onions, finely chopped
\\ 2 cloves garlic, minced
\\ 2 potatoes, diced
\\ \nicefrac{1}{2} head cauliflower, cut into florets
\\ 2 green chillies, chopped
\\ \nicefrac{1}{4} tsp turmeric powder
\\ 1 tsp salt
\\ 1 tsp garam masala
\\ \nicefrac{1}{2} tsp ginger
\\ \nicefrac{1}{2} tsp fenugreek}{Heat oil in frying pan. Add onions, mustard seeds, tomatoes, garlic and spices. Cook for 10 minutes. Stir in remaining ingredients and cover the pan, cook until the potatoes are tender.}

\end{minipage}\par\begin{minipage}{\linewidth}\rtitle{PALAK ALOO} \index{PALAK ALOO}
\step{1 lb/500 g new potatoes	
	\\ 1 bunch spinach
	\\ 2 Tbsp ghee or oil	
	\\ 1 tsp black mustard seeds
	\\ 1 tsp cumin seeds	
	\\ \nicefrac{1}{2} tsp turmeric
	\\ \nicefrac{1}{2} tsp coriander	
	\\ \nicefrac{1}{2} tsp ground cumin
	\\ 2 fresh green chillies, deseeded	
	\\ 1 tsp salt
\\ approx \nicefrac{1}{2} cup water	
\\ \nicefrac{1}{2} tsp grated nutmeg}{
Cut potatoes in small cubes. Wash spinach several times. Discard tough stems and steam in covered pot for 10 minutes, until tender. Do NOT add water. Chop roughly. Save liquid in pan.
In frying pan, heat ghee and fry mustard and cumin seeds until the mustard seeds pop. Cover the pan so that the seeds don't fly all over the stove top. Add spices and chillies. Add potatoes and fry a few more minutes. Add spinach liquid, adding water until it reaches the \nicefrac{1}{2} cup mark. Season with salt. Cover and cook 10 minutes. Add spinach, stir, cover and cook 5-10 minutes more. Sprinkle nutmeg over the dish and serve.}

\end{minipage}\par\begin{minipage}{\linewidth}\rtitle{SAAG} \index{SAAG}
\textit{Katrina: I was introduced to saag from my North Indian graduate school roommate, Jagjeet Kaur Kapoor, and it quickly became a staple in my Indian cooking. Unlike palak paneer (a spinach dish), saag has a bitterness from the mustard greens. This recipe can be turned into palak paneer by replacing the mustard greens with more spinach.}\\
	\step{\nicefrac{1}{2} tsp cumin seeds/jeera
\\ 2 small tej patta or bay leaves
\\ 1 medium onion, finely chopped
\\ 1 tsp ginger paste 
\\ 1 tsp garlic paste
\\ 1 or 2 green chilies, chopped 
\\ a pinch of turmeric powder
\\ \nicefrac{1}{2} tsp black pepper
\\ 1-2 tsp kasuri methi/dry fenugreek leaves
\\ \nicefrac{1}{2} tsp garam masala powder
\\ 1 can spinach
\\ 1 can mustard greens
\\ \nicefrac{1}{4} cup cream
\\ \nicefrac{1}{4} cup paneer
\\ salt as required}{In a small amount of oil, add cumin and bay leaves until cumin starts to smell fragrant. Add onions, ginger, garlic, and chilies. Cook until soft. Add spices and cook until fragrant. Add pre-blended spinach and mustard greens. Cook for a few minutes. Add cream and paneer.}

\end{minipage}\par\begin{minipage}{\linewidth}\rtitle{PANEER MASALA} \index{PANEER MASALA}
\textit{(Sabine Brackhahn)}\\
\step{2 onions
\\ oil or ghee
\\ 2 tsp turmeric	
\\ 1 Tbsp garam masala
\\ 1 tsp curry (more if mild curry)	
\\ 300 g can tomatoes or 8-10 fresh
\\ 1 small can of tomato paste	
\\ 1 tsp salt
\\ 1 bouillon cube (opt)	
\\ 1-2 garlic cloves (opt)
\\ 1 Tbsp chopped cilantro or 1 tsp ground coriander
\\ 15 ml cream (opt)
\\ 250 g paneer (substitutes: 250 g ricotta or 1 block of tofu}{Fry onions in oil or ghee. Add turmeric, garam masala, curry, tomatoes, tomato paste, salt, bouillon, garlic, cilantro, and cream. Simmer on low heat for 10 minutes. Add in paneer. Heat in oven for 20 minutes at 180${}^\circ$C. Do not stir because the cheese will get too crumbly.}

\end{minipage}\par\begin{minipage}{\linewidth}\rtitle{RICOTTA DUMPLINGS IN CHILLI SPINACH SAUCE} \index{RICOTTA DUMPLINGS IN CHILLI SPINACH SAUCE}
\textit{Sabine: I usually prepare all ahead and then heat up the spinach mixture, pour it in casserole, add the dumplings and then warm up in oven without stirring, for 20 minutes.\\
(Sabine Brackhahn)}\\
\step{650 g of spinach
\\ 1 big onion, chopped
\\ 2 Tbsp dried coconut	
\\ 1 Tbsp poppy seeds
\\ 4 cloves garlic, crushed	
\\ 2 green chillies, finely chopped
\\ 1 Tbsp fresh ginger pieces cut small	
\\ 1 tsp ground ginger powder
\\ \nicefrac{1}{2} piece of star anise	
\\ 1 med size onion finely chopped
\\ 2 Tbsp water
\\ 2 Tbsp ghee (or margarine)
\\ 1 cup yoghurt	
\\ 1 cup water
\\ \nicefrac{1}{3} cup coconut powder or half a can of coconut milk}{Wash and cut spinach. Fry with onion, chopped. Mix coconut, poppy seeds, garlic, chillies, ginger, ginger powder, star anise, onion and water into a paste. Fry in ghee until aromatic. Add spinach. Blend with mixer until smooth. Add yoghurt, water, and coconut powder.}
\step{Dumplings:
\\ 1 \nicefrac{1}{2} cups (300 g) ricotta cheese	
\\ \nicefrac{2}{3} cup flour
\\ \nicefrac{1}{4} cup chopped fresh coriander	
\\ 1 tsp salt
\\ \nicefrac{1}{2} tsp ground coriander}{Mix with fork or pastry cutter until well blended. Form dumplings with hands. Makes about 25. Roll in bread crumbs. Shallow-fry in small pan and drain on absorbent paper.}

\end{minipage}\par\begin{minipage}{\linewidth}\rtitle{VEGETABLE BIRYANI} \index{VEGETABLE BIRYANI}
\textit{(Neha Gupta, Katrina's cooking teacher in Delhi)}\\
\step{1 cup rice
\\ 3 tsp ghee
\\ 1 tsp cumin seeds
\\ 3 cloves
\\ 3 whole black pepper corns
\\ 1 inch cinnamon
\\ 3 green cardamom pods
\\ 2 black cardamom pods
\\ 1 mace flower
\\ 2 bay leaves
\\ 2 large onions, cut in long strips
\\ \nicefrac{1}{2} tsp garlic, minced
\\ \nicefrac{1}{2} tsp ginger, minced
\\ \nicefrac{1}{2} cup green beans
\\ \nicefrac{1}{2} cup carrots
\\ \nicefrac{1}{2} cup peas
\\ 1 tsp cumin
\\ 1 tsp coriander
\\ 1 tsp red chile
\\ 1 tsp turmeric
\\ 1 tsp garam masala
\\ \nicefrac{1}{2} tsp biryani masala
\\ 2 tsp salt
\\ 2 cups water
\\ 1 tsp ghee
\\ 1 tsp mint leaves
\\ 1 tsp cilantro
\\ \nicefrac{1}{2} tsp biryani masala
}{Rice and soak rice and set aside for 15 minutes. Add 3 tsp ghee and toast cumin seeds. Add whole spices and saut\'e until aromatic. Add ginger, garlic, and onion. Add vegetables and cook on low heat for 2 minutes. Add spices. Drain rice. Add rice and water and let come to a boil. Add remaining ghee, biryani masala, mint, and cilantro and cook uncovered until no water is left.}


\end{minipage}\par\begin{minipage}{\linewidth}\rtitle{\hypertarget{chapatilink}{CHAPATIS}} \index{CHAPATIS}
\textit{yield 4}\\
\step{1 \nicefrac{1}{2} cups flour
\\ \nicefrac{1}{2} tsp salt
\\ 1 Tbsp vegetable oil
\\ \nicefrac{1}{2} cup water, approx. (enough for an elastic dough)
\\ Vegetable oil for brushing on the chapatis}{Stir together salt and flour.  Add the 1 Tbsp of vegetable oil.  Stir in water and knead with fingers until the dough is elastic, approximately 5 minutes. Divide into 4 equal balls. Let rest several hours. Roll the ball into a thin circle. Spread \nicefrac{1}{4} tsp oil over it. Roll like a jelly roll, then coil the roll to form a snail. Let dough sit 20 minutes to 8 hours. (don't refrigerate) Coil each snail into a 10--12" diameter circle. Place on ungreased hot cast iron skilled (med heat). Flip the chapati when brown spots appear on the underside. Brush the entire surface of the chapatti lightly with oil. Flip when brown spots appear on the other side and chapati has air bubbles. Brush with oil again. Flip for a few seconds and remove from pan.}

\end{minipage}\par\begin{minipage}{\linewidth}\rtitle{BHAJIAS} \index{BHAJIAS}
\textit{A popular food in Kenya. Serve it with chutney.}\\
\step{400 g potatoes, sliced thinly
\\ 150 g chick pea flour
\\ 60 g rice flour
\\ \nicefrac{1}{2} tsp corn flour
\\ 1 tsp salt
\\ Green chilies pounded to a paste
\\ 4 large garlic cloves
\\ 6 Tbsp chopped coriander
\\ \nicefrac{1}{2} tsp carom seeds
\\ 2 \nicefrac{1}{2} tsp turmeric
\\ 1 \nicefrac{1}{2} tsp sugar}{Mix all the ingredients except the potatoes. Add water until the batter is thick enough to coat the potatoes but not clumpy. Dip potatoes in batter.  Drop into hot fat and deep fry.}

\end{minipage}\par\begin{minipage}{\linewidth}\rtitle{ONION PAKORA} \index{ONION PAKORA}
\step{2 med/large onions
\\ 1 cup gram flour
\\ 1--2 green chilies (or \nicefrac{1}{2} tsp red chili powder)
\\ 1 Tbsp chopped coriander leaves
\\ \nicefrac{1}{2} tsp garam masala powder
\\ \nicefrac{1}{4} tsp turmeric powder
\\ 1 tsp ajwain or carom seeds
\\ A generous pinch of asafetida (opt)
\\ Salt to taste}{Slice the onions thinly. Place in a mixing bowl together with the chopped chilies, coriander leaves and spices. Mix everything well. Cover and let rest for 15--20 minutes. the onions will release moisture.  Add gram flour/besan.  Add required amount of water to make a medium thick batter. Add 1--2 tsp oil to make the pakoras crisper. Stir the mixture very well with a spoon or with your hands.  Add spoonfuls of the batter to the hot oil and deep fry.  Don't overcrowd the pakoras while frying. Turn and continue to fry. Turn the pakoras several times, till they look crisp and golden. Remove with a slotted spoon and drain on paper towels. Serve the pakoras with coriander chutney.}

\end{minipage}\par\begin{minipage}{\linewidth}\rtitle{PAKORA BATTER} \index{PAKORA BATTER}
\textit{(Jagjeet Kaur Kapoor, Katrina's graduate school roommate)\\
Everything can be turned into pakora. Jeet's favourite pakora is bread pakora: bread covered with pakora batter and deep-fried. Katrina loves paneer pakora: thinly sliced paneer sandwiching a mint chutney, dipped in pakora batter, and deep-fried. Another favourite is a spinach pakora (flash-fried) drizzled with a tamarind sauce.}\\
\step{gram/chickpea flour
\\ water
\\ ajwain
\\ turmeric
\\ Tbsp hot oil
\\ salt}{This is a ``by feel'' recipe because Jeet never measures her ingredients!}


\end{minipage}\par\begin{minipage}{\linewidth}\rtitle{MIXED VEGETABLE RAITA} \index{MIXED VEGETABLE RAITA}
\step{2 cups yoghurt	
	\\ \nicefrac{1}{2} tsp salt
	\\ \nicefrac{1}{2} cup cucumber, chopped	
	\\ 1 tomato, chopped
	\\ 1 onion, chopped	
	\\ 1 green chilli, chopped
	\\ pinch cumin	pinch black pepper
	\\ 2 Tbsp coriander leaves, chopped}{}

\end{minipage}\par\begin{minipage}{\linewidth}\rtitle{MINT CHUTNEY} \index{MINT CHUTNEY}
\step{1 bunch fresh cilantro	
	\\ 1 \nicefrac{1}{2} cups fresh mint leaves
	\\ 1 medium onion, cut into chunks	
	\\ 1 green chile pepper
	\\ 1 Tbsp tamarind juice or lemon juice	
	\\ \nicefrac{1}{2} tsp salt}{
In a food processor, combine the cilantro, mint leaves, chile pepper, salt, onion and tamarind juice. Process to a fine paste.}

\end{minipage}\par\begin{minipage}{\linewidth}\rtitle{SPICY PEACH CHUTNEY} \index{SPICY PEACH CHUTNEY}
\step{4 lbs sliced, peeled peaches
\\ 1 cup raisins
\\ 2 cloves garlic, minced
\\ \nicefrac{1}{2} cup chopped onion
\\ 5 oz chopped preserved ginger
\\ 1 \nicefrac{1}{2} Tbsp chili powder
\\ 1 Tbsp mustard seed
\\ 1 tsp curry powder
\\ 4 cups packed brown sugar
\\ 4 cups apple cider vinegar
\\ \nicefrac{1}{4} cup pickling spice}{In a large heavy pot, stir together all the ingredients except the pickling spice. Wrap the pickling spice in a cheesecloth bag and place in the pot. Bring to a boil and cook over med heat uncovered until the mixture reaches your desired consistency. It will take about 1 \nicefrac{1}{2} hours to get a good thick sauce. Stir frequently to prevent scorching on the bottom. Remove the spice bag and ladle into hot sterilized jars. Wipe the rims with a clean moist cloth. Seal with lids and rings and process in a barely simmering water bath for 10 minutes. The water should cover the jars completely.}
	
\end{minipage}\par\begin{minipage}{\linewidth}\rtitle{COCONUT KULFI} \index{COCONUT KULFI}
\textit{Indian ice cream has an interesting texture that comes from the rice flour.}\\
\step{2.5 cup milk
\\ \nicefrac{1}{2} cup sweetened condensed milk
\\ \nicefrac{1}{4} cup sugar
\\ 1 tsp rice flour
\\ crushed nuts
\\ \nicefrac{1}{2} tsp cardamom
\\ cinnamon, pinch
\\ \nicefrac{1}{4} cup toasted coconut}{Mix ingredients. Simmer for 15 minutes. Store in freezer.}

\end{minipage}\par\begin{minipage}{\linewidth}\rtitle{PANEER}
(\hyperlink{paneerlink}{see Cheeses}) \\

\end{minipage}\par\begin{minipage}{\linewidth}\rtitle{GARAM MASALA} \index{GARAM MASALA}
\step{1 Tbsp cumin, ground 
\\ 1 \nicefrac{1}{2} tsp coriander
\\ 1 \nicefrac{1}{2} tsp cardamom 
\\ 1 \nicefrac{1}{2} tsp pepper
\\ 1 tsp cinnamon
\\ \nicefrac{1}{2} tsp cloves 
\\ \nicefrac{1}{2} tsp nutmeg}{Optionally toast before storing.}

\end{minipage}\par\begin{minipage}{\linewidth}\rtitle{TANDOORI MASALA} \index{TANDOORI MASALA}
\step{2 Tbsp coriander
\\ 1 \nicefrac{1}{2} Tbsp cumin
\\ 1 tsp garlic powder
\\ 1 tsp ginger
\\ 1 tsp cloves
\\ 1 tsp mace
\\ 1 tsp fenugreek
\\ 1 tsp cinnamon
\\ 1 tsp black pepper
\\ 1 tsp cardamom
\\ \nicefrac{1}{2} tsp nutmeg}{}

\end{minipage}\par\begin{minipage}{\linewidth}\rtitle{BIRYANI MASALA} \index{BIRYANI MASALA}
\step{1 bay leaf
\\ 1 \nicefrac{1}{2} tsp fennel seeds
\\ 2 star anise
\\ 10--12 green cardamom pods
\\ 2 black cardamom pods
\\ 1 tsp pepper corn
\\ 5 inch cinnamon
\\ 1 Tbsp cloves
\\ 4 Tbsp coriander seeds
\\ 2 Tbsp cumin seeds
\\ 1 Mace flower
\\ \nicefrac{1}{2} tsp nutmeg}{Toast and blend.}

\end{minipage}\par\begin{minipage}{\linewidth}\rtitle{CHAAT MASALA} \index{CHAAT MASALA}
\step{3 Tbsp cumin seeds
\\ 2 tsp ajwain
\\ 1 Tbsp ginger powder
\\ 2 tsp amchur powder
\\ 2 Tbsp black salt
\\ \nicefrac{1}{2} tsp nutmeg
\\ 1 tsp black pepper
\\ 1 tsp salt}{Toast cumin and ajwain and blend. Add remaining ingredients.}

\end{minipage}
	