\chapter{South East Asian}

\name{TIPS FOR CHINESE COOKING\\
(Taken from the Nairobi Chapel International Cookbook, contributors Kay \& Omar Djoeandy)\\
Invest in good quality soy sauce, sesame oil, dry sherry and oyster sauce. These may seem expensive at first, but as only a small amount is necessary in most dishes, they last a long time.
To prepare meat, it should be cut into small, thin strips, then marinated for tenderness and for the meat to absorb more of the flavour of the sauces and spices. This also allows it to cook quickly, thus retaining its natural juices.
Vegetables should also be cut into small pieces so that they cook quickly and retain their freshness and flavour.
Cook all on high heat with a very small amount of oil. Cook meat in small batches, allowing each piece to touch the pan directly and cook quickly rather than be stewed.}

\begin{minipage}{\linewidth}\rtitle{EGG DROP SOUP} \index{EGG DROP SOUP}
\name{(Chapel Chakula--Kay Djoeandy)} \\
\step{1 tsp fresh ginger, minced	
	\\ 1-2 cloves garlic, minced
	\\ 5 cups chicken or pork broth	
	\\ 1 Tbsp soy sauce
	\\ 1 tsp sesame oil	
	\\ 2 eggs, beaten
	\\ salt, pepper, MSG to taste	
	\\ 3-4 spring onions, chopped}{Fry ginger and garlic in 1 Tbsp oil until brown. Add broth, soy sauce and sesame oil. Let it simmer 10 minutes. Just before serving, bring to the boil and add spring onions. While stirring soup with a fork, pour beaten egg into the boiling soup in a slow, steady stream until all is added and cooked. Garnish with the green part of spring onions.}

\end{minipage}\par\begin{minipage}{\linewidth}\rtitle{MISO SOUP} \index{MISO SOUP}
\step{500 mL of water
\\ 2 Tbsp miso paste (or to taste)
\\ shredded cabbage (opt)
\\ green onions (opt)
\\ tofu (opt)
\\ seaweed (opt)}{Bring water to a boil. Add miso paste and simmer. Add optional additions to the soup. Simmer for 5 minutes.}

\end{minipage}\par\begin{minipage}{\linewidth}\rtitle{WAR WONTON SOUP} \index{WAR WONTON SOUP}
\step{\nicefrac{1}{2} med bok choy	
	\\ 2 green onions, sliced
	\\ 1 leek	
	\\ 4 med mushrooms, sliced
	\\ 4 cups chicken stock	
	\\ \nicefrac{1}{2} cup leftover roast pork, cut in 2" slivers
	\\ 8 cups water	
	\\ 32 wontons}{
Cut green, leafy tops of bok choy and chop into small pieces. But white stalks into matchstick size pieces. Put greens and white slivers into large pot. Add green onion. Cut off bottom and green part of leek and discard. Cut white part in half crosswise. Sliver into matchstick size pieces and add to pot. Add mushrooms and chicken stock. Bring to boil. Cover and simmer 20 minutes.
Boil water in large pot. Using slotted spoon, add 10 wontons at a time. When they are cooked, they will rise to the top. Transfer to soup when ready to serve. Variation:  Chicken wonton soup: Boil 12-15 wontons in 6 cups chicken stock until the float. Add 3 green onions, sliced.}

\end{minipage}\par\begin{minipage}{\linewidth}\rtitle{WONTONS} \index{WONTONS}
\step{\nicefrac{1}{2} lb lean ground pork	
	\\ 1 egg
	\\ 3 Tbsp finely chopped green onion	
	\\ 1 Tbsp soy sauce
	\\ \nicefrac{1}{4} tsp salt	
	\\ \nicefrac{1}{4} tsp monosodium glutamate (opt)
	\\ 454 grams wonton wrappers, thawed}{
Combine first 6 ingredients. If too soft to work with, add a few dry bread crumbs. Place 1 tsp filling in center of wonton wrapper. Moisten edges. Fold over, forming triangle. Press edges to seal. Moisten bottom corners, pull gently together to overlap and press together. Freeze on tray with waxed paper between layers. Store in sealed container once frozen. Makes about 3 dozen.}

\end{minipage}\par\begin{minipage}{\linewidth}\rtitle{SUKIYAKI} \index{SUKIYAKI}
\name{(Henry Hiebert)}\\
\step{1 \nicefrac{1}{2} lb. round steak, cut across grain, paper thin	
	\\ \nicefrac{1}{4} cup butter
	\\ 2 bunches green onions, cut in 2" pieces	
	\\ \nicefrac{1}{4} cup beef bouillon
	\\ 1 Spanish onion, sliced thinly 	
	\\ \nicefrac{1}{4} cup soy sauce
	\\ \nicefrac{1}{2} lb. mushrooms, sliced thin	
	\\ 1 Tbsp sugar
	\\ 1 lb. fresh spinach, shredded or 1 lb. frozen spinach
	\\ 1 (7 oz.) can bamboo shoots, sliced}{Melt butter in electric frying pan over high heat. Saut\'e beef quickly over medium heat until browned. Add bouillon mixed with soy sauce and sugar. Cover. Cook until tender on low heat, about 30 to 40 minutes. Add vegetables. Cover. Cook 10 minutes. Vegetables should be tender crisp. Serve with rice.}

\end{minipage}\par\begin{minipage}{\linewidth}\rtitle{STIR-FRIED BEEF WITH ONION} \index{STIR-FRIED BEEF WITH ONION}
\step{450 grams rump steak	
	\\ 1 Tbsp dry sherry
	\\ \nicefrac{1}{2} tsp salt	
	\\ 2 Tbsp soy sauce
	\\ 1 tsp sugar	
	\\ 4 Tbsp vegetable oil
	\\ freshly ground black pepper	
	\\ 2--3 slices root ginger, shredded
	\\ 1 large onion, thinly sliced	
	\\ 3 garlic cloves, crushed
	\\ 1 \nicefrac{1}{2} tsp cornflour, blended with 3 Tbsp stock}{Cut the steak against the grain, into thin shreds. Mix together the sherry, salt, soy sauce, sugar and 1 \nicefrac{1}{2} tsps of the oil and pepper to taste. Add steak and leave to marinate for 15 minutes.
Heat 2 \nicefrac{1}{2} Tbsp of remaining oil in wok over high heat. Add ginger and onion and stir fry for 1 \nicefrac{1}{2} minutes. Push them to one side and add remaining oil to other side of pan. When it is very hot, add the steak and garlic and stir fry for a few seconds until brown. Mix in onion and ginger and stir fry for 1 minutes. Stir in blended cornflour and cook for about 30 seconds or until thickened.}

\end{minipage}\par\begin{minipage}{\linewidth}\rtitle{STIR-FRIED BEEF WITH BROCCOLI} \index{STIR-FRIED BEEF WITH BROCCOLI}
	\step{225 grams lean beef steak, thinly sliced
	\\ 2 tsp salt	
	\\ 1 tsp dry sherry
	\\ 1 Tbsp cornflour	
	\\ 4 Tbsp vegetable oil
	\\ 225 g broccoli, broken into small florets
	\\ little chicken stock or water (opt)
	\\ 2 spring onions, cut into 1" lengths
	\\ 110 g button mushrooms, sliced
	\\ 1 Tbsp soy sauce}{
Put steak in bowl with \nicefrac{1}{2} tsp salt, sherry and cornflour. Mix well and leave to marinate for 20 minutes.
Heat 2 Tbsp of oil in wok. Add broccoli and remaining salt and stir fry for a few minutes, adding a little stock or water to moisten if necessary. Remove from pan with a slotted spoon, drain and keep on one side.
Heat the remaining oil in pan. Add spring onions and fry for a few seconds. Add the steak and stir fry until evenly browned. Stir in the mushrooms, soy sauce and broccoli.}

\end{minipage}\par\begin{minipage}{\linewidth}\rtitle{PANCIT} \index{PANCIT}
\step{1 lb package pancit bihon rice noodles
\\ \nicefrac{1}{2} lb pork sliced thin
\\ \nicefrac{1}{2} lb chicken, cooked, deboned and cut into thin slices
\\ 1 tsp vegetable oil
\\ 1 onion, finely diced
\\ 3 cloves garlic, minced
\\ 2 cups diced cooked chicken breast
\\ 1 small head cabbage, thinly sliced
\\ 1 cup celery leaves, finely chopped
\\ \nicefrac{1}{8} lb pea pods or snow peas
\\ 4 carrots, thinly sliced
\\ 1 chicken boullion cube
\\ \nicefrac{1}{4} cup soy sauce
\\  3--4 cups chicken broth
}{Place noodles in large bowl and cover with warm water. When soft, drain and set aside. Heat oil in wok over med heat. Saut\'e pork, onion and garlic. Add chicken. Add chicken cube and water. Cook for 15 minutes. Stir in vegetables. Cook until cabbage begins to soften.  Remove all ingredients with slotted spoon. Add soy sauce. Add pancit noodles and cook until most of the liquid has evaporated. Add meat and vegetables. Toss to combine.}

\end{minipage}\par\begin{minipage}{\linewidth}\rtitle{CHICKEN SATAY} \index{CHICKEN SATAY}
\step{ \nicefrac{1}{2} cup coconut milk \\
1 clove garlic, minced \\
1 tsp curry powder \\
1 \nicefrac{1}{2} tsp brown sugar \\
\nicefrac{1}{2} tsp salt \\
\nicefrac{1}{2} tsp black pepper \\
\nicefrac{3}{4} lb chicken breast, cut into 1" strips \\
1 cup coconut milk \\
1 Tbsp curry powder \\
\nicefrac{1}{2} cup creamy peanut butter \\
\nicefrac{3}{4} cup chicken stock \\
\nicefrac{1}{4} cup brown sugar \\
2 Tbsp lime or lemon juice \\
1 tsp soy sauce \\
Salt to taste \\
10 skewers}{Stir together first 6 ingredients until sugar has dissolved. Toss marinade with the chicken, cover and marinate for at least 2 hours. To make sauce, combine next 5 ingredients and simmer for 5 min, stirring constantly, until smooth and chickened. Remove from heat and stir in lime juice and soy sauce. Season with salt. Set aside.  Thread chicken onto skewers and grill 4--5 minutes per side. Serve with warm peanut sauce.}

\end{minipage}\par\begin{minipage}{\linewidth}\rtitle{GINGER PORK SAUT\'E} \index{GINGER PORK SAUT\'E}
	\step{14 oz boneless pork loin, thinly sliced	
	\\ 1 \nicefrac{1}{2} tsp grated fresh ginger
	\\ 4 Tbsp soy sauce	
	\\ 2 Tbsp sake
	\\ \nicefrac{1}{2} Tbsp sugar}{
Marinate pork for 10 minutes. Heat 3 Tbsp oil in skillet over moderate heat. Saut\'e pork until brown. Add reserved sauce. Cook pork over medium high heat, stirring until pork is glazed.}

\end{minipage}\par\begin{minipage}{\linewidth}\rtitle{SWEET AND SOUR TENDERLOIN} \index{SWEET AND SOUR TENDERLOIN}
\name{Loraine Hiebert}\\
\step{1 lb. cubed tenderloin	
	\\ egg
	\\ milk
	\\ flour
	\\ salt
	\\ oil
	\\ 1 cup pineapple chunks	
	\\ 6 sliced sweet pickles
	\\ 1 green pepper, chopped	
	\\ \nicefrac{1}{2} cup water.
	\\ 2 Tbsp vinegar	
	\\ 1 \nicefrac{1}{2} Tbsp sugar
	\\ 1 Tbsp molasses	
	\\ 1 Tbsp cornstarch
	\\ \nicefrac{1}{2} cup water}{Make batter with egg, milk, flour, and salt. Batter tenderloin and brown in oil. Add pineapple, pickles, green pepper, and water. Cook 10 minutes. Mix remaining ingredients separately. Pour over meat mixture. Cook 5 minutes.}

\end{minipage}\par\begin{minipage}{\linewidth}\rtitle{SWEET AND SOUR PORK} \index{SWEET AND SOUR PORK}
\name{(Wild Boar on the Kitchen Floor)}\\
\step{1 egg, beaten	
	\\ 1 Tbsp sugar
	\\ 1 tsp salt	
	\\ 1 Tbsp soy sauce
	\\ 1 lb pork, cubed
	\\ 1 garlic clove, crushed	
	\\ ginger root, chopped
	\\ 1 onion, chopped
	\\ 1 green pepper, chopped
	\\ \nicefrac{3}{4} cup pineapple, chopped
	\\ 3 Tbsp vinegar	
	\\ 3 Tbsp brown sugar
	\\ 2 Tbsp soy sauce	
	\\ 1 Tbsp cornstarch
	\\ \nicefrac{3}{4} cup pineapple juice}{Mix egg, sugar, salt, soysauce, and pork together. Let stand 20-30 minutes. Combine vinegar, brown sugar, soy sauce, cornstarch, and pineapple juice. Remove meat from marinade and fry. When well done, add the garlic, ginger and onion. When these are almost done, add the green pepper and pineapple. When this is almost cooked, add the sauce and let heat through thoroughly. Serve with rice.}

\end{minipage}\par\begin{minipage}{\linewidth}\rtitle{SHRIMP AND BROCCOLI IN CHILI SAUCE} \index{SHRIMP AND BROCCOLI IN CHILI SAUCE}
\step{1 \nicefrac{1}{2} lbs med shrimp, peeled and deveined	
	\\ 2 Tbsp jalapeno pepper, minced and deseeded 
	\\ 2 Tbsp dry sherry	
	\\ 1 \nicefrac{1}{2} tsp paprika
	\\ \nicefrac{1}{2} tsp ground red pepper	
	\\ 4 garlic cloves
	\\ \nicefrac{1}{3} cup water	
	\\ \nicefrac{1}{4} cup chilli sauce
	\\ 2 tsp cornstarch	
	\\ 2 tsp sugar
	\\ \nicefrac{1}{2} tsp salt	
	\\ 1 Tbsp oil
	\\ 3 cups broccoli florets	
	\\ 4 cups soba or vermicelli}{Combine first 6 ingredients in med bowl. Cover and chill for 1 hour. Combine water and next 4 ingredients in small bowl. Set aside. Heat oil in wok over med high heat. Stir fry broccoli 2 minutes. Add shrimp mixture. Stir fry 5 minutes or until shrimp are done. Add cornstarch mixture and bring to boil. Cook 1 minutes or until sauce thickens. Serve over noodles.}

\end{minipage}\par\begin{minipage}{\linewidth}\rtitle{SHRIMP AND BROCCOLI IN GARLIC SAUCE} \index{SHRIMP AND BROCCOLI IN GARLIC SAUCE}
\step{2 cups fresh broccoli florets
\\ 4 large cloves garlic, minced
\\ 3 large mushrooms, sliced
\\ 1 lb shrimp, peeled and deveined
\\ Sesame oil for stir frying
\\\\ SAUCE:
\\ 1 cup chicken broth
\\ 1 Tbsp soy sauce
\\ 1 Tbsp oyster sauce
\\ \nicefrac{1}{4} tsp ginger powder
\\ Few drop sesame oil
\\ \nicefrac{1}{2} Tbsp chili sauce
\\ 2 Tbsp cornstarch}{Stir fry garlic 30 sec. Add vegetables and stir fry. Add sauce. Allow to thicken. Add shrimp.}

\end{minipage}\par\begin{minipage}{\linewidth}\rtitle{CHINESE SHRIMP WITH GARLIC} \index{CHINESE SHRIMP WITH GARLIC}
\step{2 Tbsp oil
\\ 10 cloves garlic, chopped
\\ 1 tsp minced fresh ginger
\\ 1 8-oz can sliced water chestnuts, drained
\\ 1 cup snow peas
\\ 1 cup small white mushrooms
\\ \nicefrac{1}{4} tsp crushed red pepper flakes
\\ \nicefrac{1}{2} tsp salt
\\ 1 tsp black pepper
\\ 1 lb peeled and deveined jumbo shrimp
\\ \nicefrac{1}{2} cup chicken broth
\\ 1 Tbsp rice vinegar
\\ 2 Tbsp fish sauce
\\ 2 Tbsp dry sherry
\\ 1 Tbsp cornstarch
\\ 1 Tbsp water}{Heat oil in wok until very hot. Cook and stir garlic and ginger until fragrant, about 30 sec. Add water chestnuts, snow peas, mushrooms, red pepper flakes, salt, pepper and shrimp. Cook, stirring, until shrimp turns pink.
Combine chicken broth, rice vinegar, fish sauce and sherry in small bowl. Pour into the shrimp mixture. Cook and stir briefly to combine. Combine cornstarch and water. Stir into wok. Cook until sauce has thickened.}

\end{minipage}\par\begin{minipage}{\linewidth}\rtitle{CHICKEN CASHEW STIR-FRY} \index{CHICKEN CASHEW STIR-FRY}
\step{\nicefrac{1}{2} cup chicken broth	
	\\ 3 Tbsp oyster sauce
	\\ 1 \nicefrac{1}{2} Tbsp cornstarch	
	\\ 1 \nicefrac{1}{2} Tbsp honey
	\\ 1 Tbsp soy sauce	
	\\ 2 tsp rice wine vinegar
	\\ \nicefrac{1}{2} tsp salt	
	\\ 2 Tbsp oil, divided
	\\ 1 cup green onions, chopped
	\\ 1 small onion, cut into 8 wedges
	\\ 1 cup julienned red bell pepper	
	\\ \nicefrac{1}{2} cup diagonally sliced carrot
	\\ 1 cup sliced mushrooms	
	\\ 1 cup snow peas
	\\ 1 lb chicken thigh meat, cut into bite size pieces
	\\ \nicefrac{1}{4} cup canned pineapple chunks in juice, drained
	\\ \nicefrac{1}{3} cup cashews	
	\\ \nicefrac{1}{2} to 1 tsp crushed red pepper}{Combine first 7 ingredients in small bowl and set aside.
Heat 1 Tbsp oil over med high heat. Add \nicefrac{1}{2} of green onions and onion wedges. Stir fry 1 minutes. Add bell pepper and carrot, stir fry 2 minutes. Add mushrooms and peas, stir fry 2 minutes. Remove vegetables from pan and keep warm.
Heat 1 Tbsp oil. Add chicken. Stir fry 5 minutes. Add broth mixture, vegetable mixture, pineapple, cashews and red pepper. Bring to boil. Cook 1 minutes or until thick. Stir in \nicefrac{1}{2} cup green onions.}

\end{minipage}\par\begin{minipage}{\linewidth}\rtitle{KOREAN JAP CHAE} \index{KOREAN JAP CHAE}
\name{This is a very large recipe that will easily feed 8-10 people.}\\
\step{250-300 g. sweet potato vermicelli
\\ 350 g. beef, pork or chicken, sliced and marinated
\\ Fresh mushrooms
\\ 1 onion, sliced
\\ 1 large carrot, cut in strips
\\ 1 bunch spinach
\\ 1 egg
\\ 2 Tbsp sesame seeds, toasted
\\\\MARINADE:
\\ 1 \nicefrac{1}{2} tsp grated ginger
\\ 4 Tbsp soy sauce
\\ \nicefrac{1}{2} Tbsp sugar
\\ 2 Tbsp sherry
\\\\SAUCE:
\\ \nicefrac{3}{4} cup soy sauce
\\ 1 cup water
\\ \nicefrac{1}{2} cup sugar
\\ 2 Tbsp sesame oil
\\ 1 tsp black pepper}{Saut\'e meat and mushrooms. Set aside. Saut\'e onions and carrots with a little salt. Set aside. Blanch and drain spinach. Season with salt and sesame oil and set aside. Separate egg. Fry egg white and yolk side by side in a small pan, like a crepe. Fold in half. Cut in strips.\\
SAUCE: Bring sauce ingredients to a boil. Reduce heat to low. Add vermicelli. Stir until shine shows. Add remaining ingredients. Top with toasted sesame seeds and egg strips.}

\end{minipage}\par\begin{minipage}{\linewidth}\rtitle{INDONESIAN SATAY} \index{INDONESIAN SATAY}
\name{(Pastor Omar Djoeandy's mother)}\\
\step{1 kg of chicken, beef, lamb or pork
	\\ 2 Tbsp ketchup	
	\\ 2 Tbsp sweet soy sauce (may sub salty soy sauce with sugar)
	\\ 2 Tbsp lemon juice	
	\\ 1 Tbsp oil
	\\ 1 tsp ground cumin seed	
	\\ 2 tsp ground coriander seed
	\\ 2 cloves garlic, finely sliced	
	\\ 3 slices of ginger}{Cut meat in cubes and put on skewers. Marinate for at least 4 or 5 hours or overnight. You can also marinate meat cubes first and put on skewers before cooking. Before you place the skewers on the BBQ, dip them in a mixture of cooking oil and soy sauce so that the meat does not dry out.}

\end{minipage}\par\begin{minipage}{\linewidth}\rtitle{SESAME CHICKEN} \index{SESAME CHICKEN}
\step{3 whole boneless chicken breasts
\\ 2 Tbsp toasted sesame seeds to sprinkle on top
\\\\SAUCE:
\\ \nicefrac{1}{2} cup water
\\ 1 cup chicken broth
\\ \nicefrac{1}{4} cup rice wine vinegar
\\ \nicefrac{1}{4} cup cornstarch
\\ \nicefrac{2}{3}  cup sugar
\\ 2 Tbsp dark soy sauce
\\ 2 Tbsp sesame oil
\\ 1 tsp chili paste
\\ 1 garlic clove, minced
\\\\MARINADE:
\\ 2 Tbsp light soy sauce
\\ 1 Tbsp cooking wine or dry sherry
\\ 3 drops sesame oil
\\ 2 Tbsp flour
\\ 2 Tbsp cornstarch
\\ 2 Tbsp water
\\ \nicefrac{1}{4} tsp baking powder
\\ \nicefrac{1}{4} tsp baking soda
\\ 1 tsp vegetable oil}{Cut the chicken into 1" cubes. Mix the marinade and marinate the chicken for 20 minutes.  Prepare the sauce.  Mix all ingredients in small pot and boil until thick.  Deep fry the chicken.  Place chicken on large platter and pour the sauce over. Sprinkle with toasted sesame seeds.}

\end{minipage}\par\begin{minipage}{\linewidth}\rtitle{DICED CHICKEN WITH BROWN BEAN SAUCE} \index{DICED CHICKEN WITH BROWN BEAN SAUCE}
\step{3 Chinese dried mushrooms	
	\\ 350 grams chicken
	\\ Pinch of salt	
	\\ 2 tsp dry sherry
	\\ 2 tsp cornstarch	
	\\ 7 Tbsp oil
	\\ 1 garlic clove, crushed	
	\\ 50 g canned bamboo shoots, drained, diced
	\\ 1 green pepper, diced	
	\\ 2 Tbsp soy bean paste or hoisin sauce blended with 2 Tbsp water}{Soak mushrooms in warm water for 20 minutes. Squeeze dry and discard stalks. Cut caps into 1" cubes. Cut chicken into 1" cubes. Sprinkle with salt, sherry and cornstarch and leave to marinate for 15 minutes. Heat 5 Tbsp of oil in wok. Stir fry chicken. Transfer to a plate. Add remaining oil to pan. Stir fry garlic 1 minutes. Add mushrooms, bamboo shoots and green pepper. Stir fry for a few seconds. Add chicken with blended bean paste and stir well.}

\end{minipage}\par\begin{minipage}{\linewidth}\rtitle{TEMPURA} \index{TEMPURA}
	\step{1 cup flour	
	\\ \nicefrac{1}{2} cup cornstarch
	\\ \nicefrac{1}{2} tsp baking soda	
	\\ \nicefrac{1}{2} tsp baking powder
	\\ Water}{}

\end{minipage}\par\begin{minipage}{\linewidth}\rtitle{CHOW MEIN} \index{CHOW MEIN}
\name{(Kay Djoeandy)} \\
	\step{500 g spaghetti or fettuccini noodles	
	\\ 300 g beef, pork or chicken
	\\ 1-2 cloves garlic, minced	
	\\ 3 Tbsp soy sauce
	\\ 1 Tbsp dry sherry	
	\\ 2 tsp sesame oil
	\\ 2--3 cups vegetables lightly steamed (cabbage, carrots, celery, green beans or whatever you have)
	\\ chopped spring onions}{
Cut meat into strips and marinate at least 1 hour in 1 Tbsp soy sauce and 1 Tbsp dry sherry. Cook noodles according to package instruction and drain well. Stir fry meat in two batches in small amount of oil. Set aside and keep covered to stay warm. Fry garlic in small amount of oil. Add noodles and stir well (may add a little more oil if needed to prevent sticking). Sprinkle 2 Tbsp soy sauce and sesame oil over and mix well. Add salt, white pepper to taste. Mix in cooked meat, vegetables and spring onions.}

\end{minipage}\par\begin{minipage}{\linewidth}\rtitle{THAI NOODLES} \index{THAI NOODLES}
\name{(Val Dirks)} \\
	\step{500 g package spaghetti or thin noodles(rather less than more)
	\\ 4 cups broccoli, cut into bite size pieces	
	\\ 2 carrots cut in long thin strips
	\\ 1 Tbsp oil	
	\\ 1 Tbsp minced ginger root
	\\ 3 cloves garlic, crushed	
	\\ 340 g chicken  breasts, cut in thin strips
	\\ \nicefrac{1}{4} cup chopped, fresh cilantro (cilantro)
	\\\\SAUCE:
	\\ \nicefrac{1}{2} cup chicken stock	
	\\ 1 Tbsp vinegar
	\\ 3 Tbsp soy sauce	
	\\ 3 Tbsp peanut butter
	\\ 1 Tbsp granulated sugar	
	\\ 1 Tbsp oil
	\\ 1 \nicefrac{1}{2} tsp chilli paste or hot pepper sauce or 2 jalapenos chopped fine (opt)}{Cook noodles for 5 minutes. Add broccoli and carrots. Cook 2--3 minutes or until noodles are tender yet firm. Drain and set aside. In a large skillet, heat oil over high heat. Stir fry ginger and garlic for 30 seconds. Add chicken. Stir fry 3-5 minutes or until chicken is no longer pink inside. Add sauce to skillet and bring to boil. Remove from heat. Toss with noodles and vegetables. Sprinkle with fresh cilantro.}

\end{minipage}\par\begin{minipage}{\linewidth}\rtitle{PAD THAI} \index{PAD THAI}
\name{(Thailand's most well known noodle dish)}\\
	\step{6\nicefrac{3}{4} cups water, divided	
	\\ \nicefrac{1}{2} lb uncooked rice noodles or vermicelli
	\\ 2 Tbsp oil, divided	
	\\ \nicefrac{1}{4} cup soy sauce
	\\ \nicefrac{1}{4} cup Thai fish sauce	
	\\ 2 Tbsp brown sugar
	\\ 2 large eggs, lightly beaten	
	\\ \nicefrac{3}{4} lb skinned, boned chicken breast
	\\ 2 garlic cloves, minced	
	\\ \nicefrac{1}{2} lb med shrimp, peeled and deveined
	\\ \nicefrac{1}{2} cup 1" green onion pieces	
	\\ 2 tsp paprika
	\\ 2 cups fresh bean sprouts	
	\\ \nicefrac{1}{2} cup chopped cilantro
	\\ 2 Tbsp chopped peanuts	
	\\ 6 lime wedges}{
Cut chicken into 1" strips. Place 6 cups water in stir fry pan or wok. Bring to boil. Add noodles. Cook 4 minutes. Drain and rinse with cold water. Drain well. Place noodles in bowl. Add 1 tsp oil. Toss well. Set aside.
Combine \nicefrac{3}{4} cup water, soy sauce, fish sauce and brown sugar. Set aside.
Heat 1 tsp oil in wok over med heat. Add eggs, stir fry 1 minutes. Add eggs to noodles. Heat 1 tsp oil in pan over  med high heat. Add chicken and garlic. Stir fry 5 minutes. Add to noodle mixture. Heat 1 Tbsp oil. Add shrimp, onions and paprika. Stir fry 3 minutes. Add soy sauce mixture and noodle mixture to pan. Cook 3 minutes or until thoroughly heated. Remove from heat, toss with sprouts and cilantro. Sprinkle with peanuts. Serve with lime wedges. Yield: 6 servings.}

\end{minipage}\par\begin{minipage}{\linewidth}\rtitle{BRIOW-WAN THAI DISH} \index{BRIOW-WAN THAI DISH}
\name{(Neta Loewen)}\\
\step{clove of garlic, finely chopped
\\ 4 chicken breasts or 1 lb. pork or beef
\\ 1 large green pepper, chopped
\\ 1 large onion, chopped
\\ 1 \nicefrac{1}{2} to 2 cucumbers, peeled and chopped
\\ 1 pt cherry tomatoes, halved
\\ 1 can pineapple chunks	
\\ 4 Tbsp vinegar	
\\ \nicefrac{3}{4} tsp soy sauce
\\ 8 Tbsp sugar	
\\ 1 tsp salt
\\ 2 Tbsp flour}{Saut\'e garlic in enough oil to cover bottom of pan. Then add chopped meat and brown. Add remaining vegetables and pineapple. Cook 2 to 3 minutes stirring constantly. Mix remaining ingredients. Add sauce. Bring to boil. Add water as necessary.}

\end{minipage}\par\begin{minipage}{\linewidth}\rtitle{EGG FOO YOUNG} \index{EGG FOO YOUNG}
\step{10 eggs, well beaten	
	\\ 3 cups bean sprouts, washed and drained
	\\ 2 Tbsp chopped onion	
	\\ salt and pepper to taste
	\\ \nicefrac{3}{4} cup leftover cooked meat	 
	\\ oil
\\\\SAUCE:	
	\\ 1 Tbsp sugar	
	\\ 1 Tbsp cornstarch
	\\ 3 Tbsp soy sauce	
	\\ 1 \nicefrac{1}{2} cups water}{Add the sprouts, onion, salt and pepper and meat scraps to the beaten eggs. Heat large skillet and coat well with oil. Pour batter in hot oil and cook on both sides, like a pancake, until golden. While eggs are cooking, make the sauce. Mix the cornstarch and sugar and stir in the soy sauce and water. cook until thickened. stirring. Serve sauce over cooked eggs. Makes 6 servings.}

\end{minipage}\par\begin{minipage}{\linewidth}\rtitle{CHINESE SPICE BEEF} \index{CHINESE SPICE BEEF}
\name{(Jessie Wei)}\\
\step{1 shank beef, deboned and cut along the muscle lines
\\ Chinese spice package in a cloth bag}{Fill pressure cooker half full of water. Add 3 Tbsp soya sauce and 1 Tbsp sugar. Put spice bag and beef into pressure cooker. Bring to boil. Let beef sit in sauce overnight. In the morning, bring to a boil once more. Cool and refrigerate. Slice thinly. Serve cold.
Decorate with parsley and/or nuts. Chill. To serve, cut into thick slices.
}


\end{minipage}\par\begin{minipage}{\linewidth}\rtitle{GYOZA}  \index{GYOZA}
\name{(Alan Bolind)\\
Gyoza is a kind of dumplings originated in China. Gyoza can be steamed, fried, boiled, and more. This is a recipe to make the most poular pan-fried gyoza.  Alan Bolind, a scientist I met in Los Alamos while doing summer research at the Los Alamos National Laboratory, made these for a potluck. I fell in love with them and found out from a fellow Japanese student (an MIT student who was working with me at LANL) that this recipe is legitimate.} \\
	\step{\nicefrac{1}{3} cup chopped cabbage (boiled)	
	\\ 2 Tbsp chopped green onion
	\\ \nicefrac{1}{2} pound ground pork	
	\\ 1 tsp sesame oil
	\\ 1 tsp sugar	
	\\ 2 tsp soysauce
	\\ \nicefrac{1}{2} tsp garlic salt	
	\\ 1 tsp grated fresh ginger
	\\ 1 Tbsp vegetable oil
	\\ 20 gyoza wrappers}{
Combine all ingredients other than oil in a bowl and mix well by hands. Place a teaspoonful of filling in a gyoza wrapper and put water along the edge of the wrapper by fingers. Make a semicircle, gathering the front side of the wrapper and sealing the top. Heat oil in a frying pan. Put gyoza in the pan and fry on high heat until the bottoms become brown. Turn down the heat to low. Add \nicefrac{1}{4} cup water in the pan. Cover the pan and steam the gyoza on low heat until the water is gone.}

\end{minipage}\par\begin{minipage}{\linewidth}\rtitle{SUSHI RICE} \index{SUSHI RICE}
	\step{\nicefrac{3}{4} cup short grain rice
\\ \nicefrac{3}{4} cup plus 1 Tbsp water
\\ \nicefrac{1}{8} cup rice vinegar
\\ 1 Tbsp sugar
\\ \nicefrac{1}{2} tsp salt}{Rinse rice until water is clear. Soak at least 30 minutes.  Combine vinegar, sugar and salt.  Heat until sugar dissolves. Cook rice in water. Bring to boil. Simmer 15 minutes, covered. Let stand covered another 15 minutes. Add vinegar mixture to rice while rice is hot. Mix in wooden bowl. Stir with folding motion about 10 minutes., until rice is at room temperature.}

\end{minipage}\par\begin{minipage}{\linewidth}\rtitle{LURE THAI GREEN CURRY} \index{LURE THAI GREEN CURRY}
\name{On the way back from Houghton, we stopped by the Lure sushi restaurant and had a mussel green curry. It was delicious and once the mussels were gone, we were offered bread with a honey butter to sop up the green curry.  This recipe is the closest I could come to the green curry part.} \\
	\step{1 can coconut milk
	\\ 2 Tbsp green curry paste
	\\ 1 Tbsp brown sugar
	\\ \nicefrac{2}{3} lb shrimp
	\\ 1 Tbsp fish sauce}{
Simmer first two ingredients for 5 minutes. Add remaining ingredients and simmer for 10 minutes. To make honey butter, melt and mix about 1 Tbsp of honey with \nicefrac{1}{2} cup butter.}

\end{minipage}\par\begin{minipage}{\linewidth}\rtitle{FISH CAKE AND NORI} \index{FISH CAKE AND NORI}
\name{Katrina: I had a Korean roommate for a summer and learned this dish from her.} \\
	\step{fish cake (a white, gelatinous substance)
	\\ sesame oil
	\\ seaweed}{Slice fish cake thinly. Fry in sesame oil. Add seaweed.}


\end{minipage}\par\begin{minipage}{\linewidth}\rtitle{VIETNAMESE FRESH SPRING ROLLS} \index{VIETNAMESE FRESH SPRING ROLLS}
\step{2 oz rice vermicelli
\\ 8 rice wrappers
\\ 8 large cooked shrimp, peeled, deveined and cut in half
\\ 1 \nicefrac{1}{3} Tbsp chopped Thai basil
\\ 3 Tbsp chopped mint leaves
\\ 3 Tbsp chopped cilantro
\\ 2 leaves lettuce, chopped
\\ napa cabbage, chopped and carrots, julienned (opt)}{Bring a medium saucepan of water to boil. Boil rice vermicelli 3-5 minutes, or until al dente. Drain.
Fill a large bowl with warm water. Dip one wrapper into the hot water for 1 second to soften or 30 seconds in room temperature water. Lay wrapper flat on a damp paper towel. In a row across the center, place 2 shrimp halves, a handful of vermicelli, basil, mint, cilantro and lettuce, leave about 2 inches uncovered on each side. Fold uncovered sides inward, then tightly roll the wrapper, beginning at the end with the lettuce. Repeat with remaining ingredients. Serve with Naoc Chan and peanut sauce.}

\end{minipage}\par\begin{minipage}{\linewidth}\rtitle{NAOC CHAN} \index{NAOC CHAN}
\step{\nicefrac{1}{4} cup fish sauce
\\ 1 long fresh red chili, chopped
\\ 2 garlic cloves, minced
\\ 2 Tbsp water
\\ 2 Tbsp lime juice
\\ 1 Tbsp rice wine vinegar
\\ 1 Tbsp sugar}{}

\end{minipage}\par\begin{minipage}{\linewidth}\rtitle{PEANUT SAUCE FOR VIETNAMESE SPRING ROLLS} \index{PEANUT SAUCE FOR VIETNAMESE SPRING ROLLS}
\step{\nicefrac{3}{4} cup creamy peanut butter
\\ \nicefrac{1}{3} cup water
\\ 3 Tbsp hoisin sauce
\\ 2 Tbsp lime juice
\\ 4 \nicefrac{1}{2} tsp soy sauce
\\ 1 Tbsp sugar
\\ 2 \nicefrac{1}{4} tsp chili garlic paste
\\ 1 med clove garlic
\\ \nicefrac{1}{2} tsp sesame oil}{}

\end{minipage}
