\chapter{Soups and Salads}

\begin{minipage}{\linewidth}\rtitle{ZUPPA TOSCANA} \index{ZUPPA TOSCANA} \index{Gluten Free!soup} 
\step{1 lb of hot Italian sausage, 
\\ 3 medium to large potatoes 
\\ 1 small onion
\\ 3 strips of thick sliced bacon
\\ 4 cloves of garlic
\\ 3 cups of kale (sliced in ribbons) 
\\ 32 oz of chicken broth 
\\ 2.5 cups of water
\\ 1 \nicefrac{1}{4} cups of half \& half}{
Remove the casings from the sausage and brown in a skillet, then put aside in a stockpot. In skillet, cook bacon and onions and put in the stockpot after chopping bacon. Add sliced potatoes, chicken broth, minced garlic, and water and let boil for about 30 minutes, until the potatoes are tender. Add kale and half \& half and let simmer another 15 minutes until the kale is tender.}

\end{minipage}\par\begin{minipage}{\linewidth}\rtitle{TOMATO BISQUE} \index{TOMATO BISQUE}  
\step{2 Tbsp of butter
\\ 1 large onion, quartered \& sliced
\\ 2 Tbsp dried dill weed
\\ \nicefrac{1}{2} tsp sugar
\\ 1 tsp salt or to taste
\\ pepper to taste
\\ 1 qt whole tomatoes (2-14 \nicefrac{1}{2} oz cans)
\\ \nicefrac{1}{4} cup flour
\\ 3 \nicefrac{1}{3} cup chicken broth
\\ 1-12 oz can evaporated milk (skim is okay if you want)}{
Add butter, onion, dill, salt, pepper \& sugar to pot.  Cook until onions are soft. Add tomatoes.  Bring to a boil, break down tomatoes with a wooden spoon into smaller pieces.  Cover and simmer for 15 minutes. Add broth, reserving about \nicefrac{1}{2} cup.  Add flour to the reserved liquid and slowly add flour mixture to bisque, stirring constantly.  Simmer 5 minutes.  Add evaporated milk, continue to stir bringing bisque up to serving temperature.}


\end{minipage}\par\begin{minipage}{\linewidth}\rtitle{MULLIGATAWNY SOUP} \index{MULLIGATAWNY SOUP} \index{Gluten Free!soup} 
\name{(Rebekah Yates)\\ 
The original recipe calls for chicken (2 boneless, skinless chicken breasts cut into small pieces), which you add at the same time as the apple. We like it better with cauliflower, but it's good with chicken, too. This soup is also really good with naan, of course.} \\
\step{1 cup onion, chopped	
	\\ 2 carrots, chopped
	\\ 1 red bell pepper, chopped	
	\\ 2 Tbsp butter
	\\ 3 Tbsp flour	
	\\ 1 Tbsp curry powder
	\\ 8 cups chicken broth	
	\\ \nicefrac{1}{2}-1 large head of cauliflower, broken/cut into small pieces
	\\ 1 apple, chopped	
	\\ \nicefrac{1}{2} cup rice (uncooked) or 1 cup cooked rice
	\\ salt to taste	ground black pepper to taste
	\\ a few shakes of dried thyme	
	\\ half \& half to pour on top}{
Brown onions, pepper, and carrots in the butter. Add flour and curry powder, and cook until it looks like it's getting too dry. Add chicken broth a bit at a time to get to 5 minutes, mix well, and bring to a boil. Add the cauliflower and simmer about \nicefrac{1}{2} hour.  Add apple, salt, pepper, thyme, and rice if you're using uncooked rice. Simmer 15--20 minutes (until the uncooked rice is done), or 10--15 minutes if you're going to use pre-cooked rice (we always use pre-cooked since we eat brown rice and it takes an hour at least). If you're using pre-cooked rice, add it now and keep simmering until the rice is heated. Serve with a bit of half-and-half poured on top.}

\end{minipage}\par\begin{minipage}{\linewidth}\rtitle{TORTILLA SOUP} \index{TORTILLA SOUP} \index{Gluten Free!soup} 
\step{2 4-oz chicken breasts, cubed	
	\\ 2 cups whole kernel corn
	\\ 1 large onion, chopped	
	\\ 2 cloves garlic, pressed
	\\ 3 \nicefrac{1}{2} cups chicken broth	
	\\ 10 \nicefrac{3}{4} oz can tomato puree
	\\ 10 \nicefrac{3}{4} oz can diced tomatoes with green chillies
	\\ 1 tsp salt	
	\\ 2 tsp ground cumin
	\\ 1 tsp chilli powder	
	\\ \nicefrac{1}{8} tsp ground red pepper
	\\ \nicefrac{1}{8} tsp ground black pepper	
	\\ 1 bay leaf
	\\ taco chips for garnish}{}
	
\end{minipage}\par\begin{minipage}{\linewidth}\rtitle{ITALIAN CHILI} \index{ITALIAN CHILI} \index{Gluten Free!soup} 
\name{(Rebekah Yates)\\
Adapted from Pampered Chef Lazy Lasagna Chili} \\
\step{1 cup zucchini, chopped (opt.)	
	\\ 1 bell pepper, chopped
	\\ 1 large onion, chopped	
	\\ 1-2 links sweet Italian sausage
	\\ 3 garlic cloves, chopped or pressed	
	\\ 1 tsp oregano	
	\\ 1 tsp basil	
	\\ 1-2 tsp sugar
	\\ 1 jar spaghetti sauce	
	\\ 4 cups beef broth
	\\ 1 cup water	
	\\ 1 cup pasta (I usually use shells)}{
Remove casings from sausage and brown the sausage, onions, pepper, and garlic. Drain any excess fat. Add spaghetti sauce, broth, sugar, oregano, basil, and water, and bring to a boil. Stir in pasta. Reduce heat, and simmer uncovered for about 7 minutes. Add the zucchini and simmer for 3-5 minutes or until zucchini is desired tenderness (I sometimes also add fresh/frozen spinach sometime in here). Top with parmesan.}


\end{minipage}\par\begin{minipage}{\linewidth}\rtitle{BLACK BEAN SOUP}  \index{BLACK BEAN SOUP} \index{Gluten Free!soup} 
\name{(Rebekah Yates)\\
adapted from Betty Crocker} \\
\step{1 large onion, chopped	
\\ 3 cloves of garlic, minced
\\ 2 \nicefrac{2}{3} cups dried black beans (1 pound)	
\\ 3 cups beef broth
\\ 3 cups water	
	\\ \nicefrac{1}{4} cup dark rum or apple cider (juice works, like grapefruit or cranberry)
	\\ 1 \nicefrac{1}{2} tsps ground cumin	
	\\ 1 \nicefrac{1}{2} tsps dried oregano
	\\ 1 bell pepper, chopped	
	\\ 1 large tomato, chopped}{
Brown the onion and garlic together. Stir in remaining ingredients; heat to boiling. Boil 2 min; reduce heat. Cover and simmer about 2 hours or until beans are tender (it can take a lot longer than two hours, depending on your beans--one way to make it faster is to add the tomatoes after 1 \nicefrac{1}{2} hours or so instead of right away). Serve with lime, cheese, and sour cream. 
If you're making it in a crockpot, decrease the water to 2 cups and just thourow everything in right away. Cook on high for 6-10 hours.}

\end{minipage}\par\begin{minipage}{\linewidth}\rtitle{WHITE CHICKEN CHILI} \index{WHITE CHICKEN CHILI} \index{Gluten Free!soup}
\name{(Laura Haines)} \\
\step{1 lb boneless skinless chicken breast (3-4) diced in cubes
\\ 1 med onion diced	
\\ 2 cloves of garlic
\\ 1 Tbs oil	
\\ 1 large (40 oz) can Great Northern beans
\\ 1 can Chicken broth	
\\ 1 tsp salt
\\ 1 tsp cumin	
\\ 1 tsp oregano
\\ \nicefrac{1}{2} tsp pepper	
\\ \nicefrac{1}{4} tsp cayenne pepper
	\\ 1 cup sour cream	
	\\ \nicefrac{1}{2} cup whipping cream}{
Brown chicken in pan. Saut\'e onions and garlic in oil. Add all ingredients except sour cream and whipping cream to the crockpot. Simmer. Add sour cream and whipping cream before serving.}

\end{minipage}\par\begin{minipage}{\linewidth}\rtitle{PUMPKIN SOUP} \index{PUMPKIN SOUP} \index{Gluten Free!soup}
\name{(Company's Coming)} \\
\step{2 Tbsp butter	
	\\ \nicefrac{1}{2} cup chopped onion
	\\ 2 cups chicken broth
	\\ 1 good size potato
	\\ \nicefrac{1}{2} tsp salt
	\\ \nicefrac{1}{4} tsp pepper
	\\ \nicefrac{1}{8} tsp thyme
	\\ 14 oz. cooked, mashed pumpkin	
	\\ 2 cups milk}{
Saut\'e onions in butter. Do NOT brown. Add remaining ingredients except pumpkin and milk. Bring to boil. Simmer until potato is tender. Run through blender. Stir in pumpkin and milk.}

%\end{minipage}\par\begin{minipage}{\linewidth}\rtitle{CURRIED PUMPKIN SOUP} \index{CURRIED PUMPKIN SOUP} \index{Gluten Free!soup}
%\step{2 Tb butter	
	%\\ 1.5 cups half-and-half cream
	%\\ 3 Tbsp flour	
	%\\ 2 Tbsp soy sauce
	%\\ 2 Tbsp curry powder	
	%\\ 1 Tbsp sugar
	%\\ 4 cups vegetable broth	
	%\\ salt and pepper to taste
	%\\ 1 (29 oz) can of pumpkin}{
%Melt butter in a large pot over medium heat. Stir in flour and curry powder until smooth. Cook, stirring, until mixture begins to bubble. Gradually whisk in broth, and cook until thickened. Stir in pumpkin and half-and-half. Season with soy sauce, sugar, salt, and pepper. Bring just to a boil, then remove from heat.}

\end{minipage}\par\begin{minipage}{\linewidth}\rtitle{GERMAN HOT POTATO SALAD}  \index{GERMAN HOT POTATO SALAD}
\name{(Elke Muller, Nebobongo Dr's wife)} \\
\step{1 \nicefrac{1}{2} lb potatoes, sliced
	\\ 6 strips bacon, chopped
	\\ 1 med onion, chopped
	\\ \nicefrac{1}{3} cup vinegar	
	\\ \nicefrac{1}{2} cup water
	\\ \nicefrac{1}{2} cup chopped parsley	
	\\ 2 tsp flour
	\\ 3 tsp sugar	
	\\ 1 \nicefrac{1}{2} tsp salt
	\\ \nicefrac{1}{4} tsp pepper}{
Boil potatoes. Cook bacon. Cook onions in some of the bacon grease. Add remaining ingredients. Cook slightly until thick. Add potatoes and blend well. Serve warm.}

\end{minipage}\par\begin{minipage}{\linewidth}\rtitle{TACO SALAD} \index{TACO SALAD}
\name{(Brenda Koop)} \\
\step{\nicefrac{1}{2} large bottle French dressing	
	\\ \nicefrac{1}{2} large bottle thousand island dressing
	\\ 1 envelope taco seasoning	
	\\ 1 lb. ground beef, browned with onions
	\\ 1 tin kidney beans	
	\\ shredded cheddar cheese
	\\ 2 tomatoes	
	\\ lettuce (less than a head.)
\\ Tortilla chips}{}

\end{minipage}\par\begin{minipage}{\linewidth}\rtitle{BOILED COLESLAW} \index{BOILED COLESLAW} \index{Gluten Free!salad}
\name{(Shirley Lysack)} \\
\step{1 fair sized head of cabbage	
	\\ \nicefrac{1}{4} green pepper
	\\ \nicefrac{3}{4} chopped onion	
	\\ 2 carrots, shredded}{Pour dressing over vegetables and mix well. Should marinate for a while before serving. Celery seeds give a nice change, also. This recipe will keep for a month in the fridge.}
	
Boil dressing: \\
\step{1 cup vinegar	
	\\ 1 cup sugar
	\\ 2 tsp salt	
	\\ \nicefrac{3}{4} cup salad oil
	\\ 1 clove garlic.}{} 

\end{minipage}\par\begin{minipage}{\linewidth}\rtitle{QUINOA SALAD WITH BLACK BEANS} \index{QUINOA SALAD WITH BLACK BEANS} \index{Gluten Free!salad}
\name{(Laura Jackson)} \\
\step{1 cup dry quinoa, rinsed	
	\\ 1 Tbsp olive oil or coconut oil
	\\ 1 \nicefrac{3}{4} cup water	
	\\ 1 can black beans, drained and rinsed
	\\ 1 avocado, chopped into chunks	
	\\ handful cherry tomatoes, quartered 
	\\ \nicefrac{1}{4} red onion, diced	
	\\ 1 small clove garlic, minced
	\\ 1 red bell pepper, chopped
	\\ handful cilantro, diced
	\\ 1 limes, juiced	
	\\ \nicefrac{1}{2} tsp  cumin
	\\ \nicefrac{1}{2} Tbsp olive oil 	
	\\ salt, to taste}{
Warm the olive/coconut oil in a medium saucepan over medium heat. Once it's hot add the rinsed quinoa and toast for about 2--3 minutes until it starts smelling nutty. Add water, stir once, cover, and simmer with a lid on for 20 minutes.  Combine the lime juice, oil, cumin, and salt. When the quinoa has finished cooking, remove it from heat and fluff with a fork. Add black beans and toss to warm them through. Let the quinoa cool for about five minutes and then add all the remaining ingredients, including the dressing, and mix.}

\end{minipage}\par\begin{minipage}{\linewidth}\rtitle{SPINACH BACON SALAD} \index{SPINACH BACON SALAD}
\name{(Beth Koehler)} \\
\step{6 oz spinach	
	\\ \nicefrac{1}{2} cup shaved parmesan
	\\ 3 hard-boiled eggs, coarsely chopped	
	\\ 6 bacon slices, crumbled
	\\ Mushourooms, sliced	
	\\ onion greens, chopped}{}
	
Dressing: \\
\step{\nicefrac{1}{2} cup spinach	
	\\ 1 \nicefrac{1}{2} tsp dried parsley
	\\ 2 bacon slices, crumbled	
	\\ 1 Tbsp Dijon mustard
	\\ \nicefrac{1}{3} cup yogurt	
	\\ 1 garlic clove
	\\ \nicefrac{1}{4} cup oil	
	\\ \nicefrac{1}{4} tsp pepper
	\\ 2 Tbsp white wine vinegar	
	\\ 2 tsp sugar}{}
	
\end{minipage}\par\begin{minipage}{\linewidth}\rtitle{CAESAR SALAD DRESSING} \index{CAESAR SALAD DRESSING}
\name{(Loraine Hiebert)} \\
\step{\nicefrac{1}{2} cup olive oil	
	\\ \nicefrac{3}{4} tsp salt
	\\ \nicefrac{1}{4} tsp pepper	
	\\ \nicefrac{1}{4} tsp dry mustard
	\\ 1 \nicefrac{1}{2} tsp Worcestershire	
	\\ 1 egg
	\\ 2 Tbsp grated parmesan cheese	
	\\ 2 Tbsp lemon juice
	\\ garlic to taste}{}

\end{minipage}\par\begin{minipage}{\linewidth}\rtitle{HARE KRISHNA SALAD DRESSING} \index{HARE KRISHNA SALAD DRESSING}
\name{(Fiore Carpino)} \\
\step{1-\nicefrac{1}{2} cups almonds	
	\\ 1-\nicefrac{1}{2} cups nutritional yeast
\\ 2 cups water	
\\ 1-\nicefrac{1}{2} cups sunflower oil
\\ \nicefrac{1}{2} cup Bragg's amino acids (or soy sauce)
\\ 1 tsp hing/asafetida}{}
		
\end{minipage}\par\begin{minipage}{\linewidth}\rtitle{CILANTRO TACO DRESSING} \index{CILANTRO TACO DRESSING}
\step{1 cup loosely packed cilantro	
	\\ \nicefrac{1}{2} cup yogurt
	\\ 2 cloves of garlic	
	\\ 1 lime
	\\ \nicefrac{1}{4} cup olive oil	
	\\ 2 Tbsp apple cider vinegar
	\\ salt}{}
	\end{minipage}
	