\chapter{Mennonite/Eastern European}

\begin{minipage}{\linewidth}\rtitle{BAKED PYROHY (PRIZHKY)} \index{BAKED PYROHY (PRIZHKY)}
\textit{(Lorraine Hiebert)} \\
\step{1 pkg. yeast	
	\\ 1 tsp sugar
	\\ \nicefrac{1}{2} cup water
	\\ 1 cup milk, scalded
	\\ 1 tsp sugar	
	\\ \nicefrac{1}{2} cup warm water
	\\ \nicefrac{1}{3} cup oil	
	\\ 2 eggs, beaten
	\\ 1 tsp salt	
	\\ 2 \nicefrac{1}{2} cups flour
	\\ 2 \nicefrac{1}{2} cups flour}{
Dissolve yeast and sugar in water. Let stand 10 min and add yeast to rest of ingredients, except final flour.
Beat thoroughly and gradually add final 2 \nicefrac{1}{2} cups more of flour. Knead well. Cover. Let stand in warm place to rise until double in bulk. (1 hour or so). Roll out. Place filling on dough. I use cookie cutter to make my pyrohy. Let rise. Bake at 350$^\circ$F for 25 min. Makes 4-5 doz. Serve with sour cream.}

\step{FILLING:
	\\ 1 med. onion, chopping fine	
	\\ 4 Tbsp butter
	\\ 1 lb. ground beef or half pork/half beef
	\\ salt and pepper	
	\\ 1 Tbsp flour
	\\ \nicefrac{1}{2} cup soup stock of water	
	\\ 1 tsp chopped parsley
	\\ 2 hard cooked eggs, chopped}{
Cook onions in half the butter until tender. Add remaining butter (Mom doesn't) and meat. Brown meat lightly. Season with salt and pepper. Cover and cook over low heat until done. Remove meat. Stir flour into drippings. Add soup stock then cook, stirring until sauce comes to boil. Combine with meat. Cool. Mix in parsley and chopped eggs.}

\end{minipage}\par\begin{minipage}{\linewidth}\rtitle{COTTAGE CHEESE WARENEKI} \index{COTTAGE CHEESE WARENEKI}
\step{2 cups of flour
	\\ \nicefrac{1}{4} cup milk	
	\\ \nicefrac{1}{4} cup sweet cream
	\\ 2 egg whites	
	\\ pinch of salt
	\\ 1 lb cottage cheese
	\\ 2 egg yolks
	\\ cream}{
Sift 2 cups of flour with the salt, make well in flour, and add milk, cream, egg, and salt. Add enough flour until you have a nice soft dough. Roll out, fairly thin, cut in rounds. Fill with cottage cheese. To cottage cheese, add egg yolks, a little cream and salt to taste. Mix well with hands, put 1 rounded tsp on each circle. Fold over in half and pinch together. Cook in salted boiling water for 5 min. Serve with cream gravy. Cream Gravy: Melt 2 or 3 Tbsp butter in frying pan, allow to cook until slightly browned, add 1 cup sweet cream. Bring to a boil.}

\end{minipage}\par\begin{minipage}{\linewidth}\rtitle{WARENIKI (GF)} \index{WARENIKI (GF)} \index{Gluten Free!wareniki}
\textit{This dough can be used to make noodles as well.}\\
\step{\nicefrac{1}{2} cup cream style cottage cheese
\\ 1 egg
\\ \nicefrac{1}{4} cup milk
\\ 1 Tbsp oil
\\ \nicefrac{1}{2} cup brown rice flour
\\ \nicefrac{1}{4} cup potato starch
\\ \nicefrac{1}{4} cup tapioca starch
\\ \nicefrac{1}{4} cup cornstarch
\\ 1 tsp xanthan gum
\\ \nicefrac{1}{2} tsp salt}{Blend first 4 ingredients until totally smooth.  Add the dry ingredients and mix thoroughly. Knead until smooth and dough is non-sticky.  Use either sweet rice flour or tapioca flour for rolling.
Tip:  Always freeze your wareniki before cooking.  They keep the seal better.  Boil for about 5 min.}

\end{minipage}\par\begin{minipage}{\linewidth}\rtitle{UKRAINIAN PEROGIES} \index{UKRAINIAN PEROGIES}
\step{4 cups flour	
	\\ 1 tsp butter or lard
	\\ 1 \nicefrac{1}{2} cup warm water	
	\\ 1 tsp salt
\\FILLING: \\	
	\\8 med. Potatoes	
	\\ 1 cup cottage cheese or \nicefrac{1}{2} cup cheese whiz
	\\ salt and pepper	
	\\ fried onions done up with butter}{
Mix ingredients to a soft dough. Let stand for awhile. Roll out dough; cut in circles about the size of an orange. Place 1 tsp of filling on each round. Fold and pinch edges together. Be sure they are sealed. Place on floured tea towel. Do not let perogies touch as they will stick. Drop in boiling salted water for 8--10 min. They will rise to the top when they are cooked. Pour melted butter over cooked perogies, so they will not stick. A good hint also is to add some lard to your boiling water to prevent sticking.}

\end{minipage}\par\begin{minipage}{\linewidth}\rtitle{CABBAGE ROLLS (HOLUBSCHI)} \index{CABBAGE ROLLS (HOLUBSCHI)}
\step{1 large head cabbage
\\ 1 \nicefrac{1}{2} lbs hamburger
\\ 1 cup cooked rice
\\ 1 med onion, chopped and Saut\'ed
\\ Salt and pepper to taste
\\ 1 large tin tomato sauce}{Place cabbage in boiling water and peel off leaves as they become tender.  Set aside.  Combine onions, meat, rice and seasoning. Blend well. Place a spoonful of meat on a cabbage leaf. Roll up and secure. Place in greased casserole. Repeat until all the meat mixture has been used.  Pour tomato sauce over all. Bake in 350${}^\circ$F oven for 2 hrs or until the cabbage is very tender.}

\end{minipage}\par\begin{minipage}{\linewidth}\rtitle{WURST BUBBAT} \index{WURST BUBBAT}
\textit{(Mennonite Treasury)}\\
\step{1 pkg. yeast	\\
3 cups milk\\
\nicefrac{1}{2} cup butter\\	
5 cups flour\\
2 cups raisins\\	
1 \nicefrac{1}{2} tsp salt}{Mix as for buns. Line bottom of deep pan with sausage or other meat. Cover with batter. Place more sausage on top of batter. Let rise in warm place for 1 hour. Bake for 1 hour in 375$^\circ$F oven.}

\end{minipage}\par\begin{minipage}{\linewidth}\rtitle{MENNONITE CHICKEN STUFFING (BUBBAT)} \index{MENNONITE CHICKEN STUFFING (BUBBAT)}
 \textit{(Loraine Hiebert)}\\
\step{1 cup flour	\\
	1 \nicefrac{1}{2} tsp baking powder\\
	\nicefrac{1}{2} tsp salt	\\
	2 Tbsp shortening melted\\
	1 egg (beaten)	\\
	\nicefrac{1}{3} cup milk\\
	1 cup raisins	\\
	2 Tbsp sugar if desired}{
Mix dry ingredients. Add egg, shortening and milk. Mix well, add raisins. Bake 350${}^\circ$F for 20 minutes or use as a filling in chicken.}


\end{minipage}\par\begin{minipage}{\linewidth}\rtitle{BEET BORSCHT} \index{BEET BORSCHT}
\step{\nicefrac{1}{2} cup pared carrots	
	\\ 1 cup skinned onions
	\\ 2 cups pared beets
	\\ 1 Tbsp butter	
	\\ 2 cups beef stock
	\\ 1 cup very finely shredded cabbage	
	\\ 1 Tbsp vinegar}{
Chop carrot, onions, and beets very finely. Barely cover these ingredients with boiling water. Simmer gently, covered, about 20 min. Add final ingredients and simmer 15 more min. Season to taste. Place the soup in bowls. Add to each serving: 1 Tbsp cultured sour cream}

\end{minipage}\par\begin{minipage}{\linewidth}\rtitle{SCHMOOR KOHL} \index{SCHMOOR KOHL}
\step{12 cups chopped white cabbage	\\
2 cups chopped dried apples\\
15 prunes\\
\nicefrac{3}{4} cups water	\\
\nicefrac{1}{3} cup cooking oil\\
2 \nicefrac{1}{2} tsp salt	\\
\nicefrac{1}{3} cup white sugar}{Cook uncovered in a heavy saucepan for several minutes. Cover and simmer 1 \nicefrac{1}{2} to 2 hours. If the fruit is very sour, add more sugar to taste.}


\end{minipage}\par\begin{minipage}{\linewidth}\rtitle{ROLLKUCHEN} \index{ROLLKUCHEN}
\textit{Always to be eaten with lots of watermelon!!!} \\
	\step{4 tsp baking powder	
	\\ 6 eggs
	\\ 2 cups sour milk	
	\\ 2 tsp salt
	\\ 1 cup butter}{
Mix dry ingredients. Add melted butter to milk. Enough flour to roll out. Knead a little but handle dough as little as possible. Cut into long strip, approx. 5 by 2 inches. Make a slit lengthwise. Fry in hot fat as you would doughnuts.}

\end{minipage}\par\begin{minipage}{\linewidth}\rtitle{ROLLKUCHEN (GF)} \index{ROLLKUCHEN (GF)} \index{Gluten Free!rollkuchen} 
\textit{(EMC Convention in High Level, AB)}\\
\step{1 cup milk
\\ 1 cup cream
\\ 4 eggs
\\ 2 Tbsp salt
\\ 1 tsp baking powder (heaping)
\\ Celiemix bread mix, enough to make a soft dough}{}

\end{minipage}\par\begin{minipage}{\linewidth}\rtitle{PORZELKY} \index{PORZELKY}
\textit{(Loraine Hiebert)}\\
\step{2 pkg yeast	
\\ 2 cups warm water
\\ 5 tsp sugar
\\ 2 cups flour
\\ 1 \nicefrac{1}{3} cups milk, scalded \& cooled	
\\ 2 tsp salt
\\ 2 tsp baking soda	
\\ 4 eggs
\\ 1 lb raisins	
\\ approx. 4 cups flour (enough to make a stiff dough)}{Let yeast, water, and sugar stand for 15 minutes. Add flour to make sponge. When light, add the remaining ingredients. Let the dough rise until double in bulk. Drop spoonfuls into hot oil. Deep fry. Yield: 7 dozen}

\end{minipage}\par\begin{minipage}{\linewidth}\rtitle{PORZELKY (GF)} \index{PORZELKY (GF)} \index{Gluten Free!porzelky}
\step{2 cups warm milk
\\ 2 Tbsp unsalted butter, softened
\\ 1 tsp sugar
\\ 1 Tbsp rapid rise yeast
\\ 2 large eggs
\\ 1 Tbsp oil
\\ 1 tsp vanilla extract
\\ 2 \nicefrac{1}{4} cups brown rice flour
\\ \nicefrac{3}{4} cup tapioca starch
\\ \nicefrac{1}{2} cup potato starch
\\ \nicefrac{1}{4} cup granulated sugar
\\ 2 Tbsp dry milk powder or almond meal
\\ 2 tsp xanthan gum
\\ \nicefrac{1}{2} tsp salt
\\ 2 cups raisins}{Add the butter to the warm milk. Stir until the butter is melted. Add 1 tsp sugar and the yeast. Stir and let proof for 5 min, until it is frothy.
While the yeast is proofing whisky together all the dry ingredients in the bowl of your stand mixer. Add the eggs, oil and vanilla extract to the yeast mixture. With the mixer on low speed, pour in the milk/egg mixture. Mix on medium speed for about 1 minute. Add raisins and stir until evenly distributed.
Place the bowl in a draft-free warm place and let rise until double in size (about 30 min).
Heat oil in thick-bottomed pot. Carefully drop 1-2 Tbsp of dough into the hot oil. Fry until golden brown on each side. Drain on paper towel. Can be rolled in icing sugar.}


\end{minipage}\par\begin{minipage}{\linewidth}\rtitle{PLAUTSE} \index{PLAUTSE}
\textit{(Mrs. Harry Fast)\\
This is one of those desserts that shows up at every funeral.} \\
	\step{3 cups flour		
	\\ \nicefrac{3}{4} cup lard
	\\ 3 tsp baking powder	
	\\ 1 beaten egg
	\\ 1 tsp salt}{
To beaten egg, add enough milk to equal 1 cup. Place dough into cookie sheet. Cover with fruit. If fruit is sour, sprinkle with sugar.}

Topping for Plautse: \\
	\step{3 Tbsp butter	
	\\ 1 \nicefrac{1}{4} cups flour
	\\ \nicefrac{1}{2} tsp baking powder	
	\\ 1 cup sugar}{Cream. Drizzle plautse with icing sugar.}

\end{minipage}\par\begin{minipage}{\linewidth}\rtitle{PLUMI MOOS} \index{PLUMI MOOS}
\textit{This is a "winter" dish. We eat this during the meal, but it's a sweet (dessert?) soup. I like it both cold and warmed.} \\
	\step{5 qt. water		
	\\ 2 cups plums
	\\ \nicefrac{1}{2} cup apricots	
	\\ 1 cup dried apples
	\\ 1 \nicefrac{1}{2} cups raisins	
	\\ 1 \nicefrac{1}{2} cups sugar
	\\ \nicefrac{1}{2} lemon		
	\\ \nicefrac{1}{2} tsp cinnamon
	\\ \nicefrac{1}{2} cup cornstarch}{
Boil water and plums 10 min. Add the remainder of the fruit and boil until well done. Then add sugar. Mix cornstarch in a small amount of water and boil for about 5 min.}

\end{minipage}\par\begin{minipage}{\linewidth}\rtitle{PASKA} \index{PASKA}
	\step{3 pkgs. yeast	
	\\ 4 cups sugar
	\\ \nicefrac{1}{2} cup lard	
	\\ 16 eggs
	\\ 1 tsp salt	
	\\ \nicefrac{1}{2} cup butter
	\\ juice and rind of 1 lemon	juice 
	\\ rind of 1 orange
	\\ 3 cups scalded milk	
	\\ 8 cups flour}{
Soften yeast in 1 cup warm water and 2 tsp sugar. Beat eggs well. Add sugar gradually and beat until dissolved. Sift flour and salt. Make well in flour, put in scalded milk. Stir and add egg mixture. Beat well. Add softened butter and fruit juices and rind. Now add yeast mixture and knead as you do sweet buns, adding flour as you need it. Let this rise in a warm place. When light, punch and let rise until double in bulk. Grease honey pails and fill \nicefrac{1}{3} with dough. Let rise until light and bake in 325$^\circ$F oven for 1 hour. Yield: 12 paska}

\end{minipage}\par\begin{minipage}{\linewidth}\rtitle{CHEESE SPREAD FOR PASKA} \index{CHEESE SPREAD FOR PASKA}
	\step{4 cups cottage cheese	
	\\ yolks of 9 hard-cooked eggs
	\\ 1 cup cream	
	\\ 1 cup butter
	\\ 1 cup sugar	
	\\ 1 tsp grated lemon rind
	\\ 1 Tbsp lemon juice}{Press cheese and egg yolks through a sieve. Bring cream to a boil and then cool. Cream together the butter and sugar and add the other ingredients. Mix thoroughly. This is now ready to use as a spread when serving Paska.}

\end{minipage}