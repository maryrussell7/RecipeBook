\chapter{Mediterannean and Middle Eastern}

\begin{minipage}{\linewidth}\rtitle{GREEK SALAD} \index{GREEK SALAD} \index{Gluten Free!salad}
\name{(Loraine Hiebert)} \\
\step{lettuce (Romaine)	
	\\ peppers
	\\ tomatoes	
	\\ red onion
	\\ whole ripe olives	
	\\ feta cheese
	\stepspace DRESSING:
  \\ 6 Tbsp olive oil	
	\\ 2 Tbsp lemon juice
	\\ 1 tsp coarse ground oregano or \nicefrac{1}{2} tsp finely ground oregano
	\\ 1 clove garlic, minced	
	\\ salt and pepper to taste}{}

\end{minipage}\par\begin{minipage}{\linewidth}\rtitle{GREEK PASTA SALAD} \index{GREEK PASTA SALAD}
\name{(Auntie Marion Hiebert)} \\
\step{1 lb spinach fettuccini	
	\\ \nicefrac{1}{2} cup chicken or veg broth
	\\ 2 Tbsp olive oil	
	\\ 2 Tbsp balsamic or white wine vinegar
	\\ 3 cloves garlic, minced	
	\\ 1 tsp dried basil
	\\ 1 tsp dried oregano	
	\\ 4 oz feta cheese, crumbled
	\\ 8 oz spinach, chopped	
	\\ 1 cucumber, peeled and chopped
	\\ \nicefrac{1}{2} small red onion, very thinly sliced
	\\ black pepper and salt	
	\\ 10 cherry tomatoes, quartered}{}

\end{minipage}\par\begin{minipage}{\linewidth}\rtitle{COUSCOUS SALAD} \index{COUSCOUS SALAD}
\name{(Don Maghee)\\
I served this once at a Birthday party in Nairobi. It was the typical Nairobi event with many nationalities represented. One of the guests was obviously from the middle east. He regarded my offering of couscous salad with great suspicion. I had great confidence in my Armenian friend's recipe and so I watched with interest to see what the guest would do. After gingerly sampling the salad, I heard him make a surprised grunt of approval and throughout the evening he kept going back for one more helping. Mr. Maghee's recipe obviously met with his approval.}\\
	\step{2 cups couscous, cooked and cooled	
	\\ 1 cup peeled seeded, chopped cucumbers
	\\ 1 cup chopped tomato	
	\\ 1 med onion, minced
	\\ \nicefrac{1}{4} cup chopped
	\\ parsley
\stepspace 	DRESSING:
	\\ \nicefrac{1}{3} cup oil	
	\\ 2 Tbsp red wine vinegar or lemon juice
	\\ 1 tsp Dijon mustard OPT	
	\\salt and pepper to taste}{Combine couscous, cucumbers, tomatoes, onion, and parsley. Pour dressing over couscous mixture. Toss gently. Chill. Serve on lettuce leaves.}

\end{minipage}\par\begin{minipage}{\linewidth}\rtitle{MEDITERRANEAN CHOPPED SALAD} \index{MEDITERRANEAN CHOPPED SALAD}
	\step{1 cup tomato, diced	
	\\ \nicefrac{1}{2} cup diced, peeled cucumber
	\\ 2 Tbsp chopped green onions	
	\\ 2 Tbsp chopped fresh cilantro
	\\ 2 tsp minced seeded jalapeno pepper	
	\\ 2 tsp fresh lemon juice
	\\ 2 tsp olive oil	
	\\ \nicefrac{1}{4} tsp salt
	\\ \nicefrac{1}{8} tsp black pepper}{}

\end{minipage}\par\begin{minipage}{\linewidth}\rtitle{TANGY LENTIL SALAD} \index{TANGY LENTIL SALAD}
\step{2 cups water	
	\\ 1 cup dried lentils
	\\ \nicefrac{1}{2} tsp salt	
	\\ 1 bay leaf
	\\ 1 cup diced, seeded, peeled cucumber	
	\\ \nicefrac{1}{2} cup diced celery
	\\ \nicefrac{1}{4} cup diced red onion	
	\\ \nicefrac{1}{4} cup orange juice
	\\ 2 Tbsp white wine vinegar	
	\\ 4 tsp Dijon mustard
	\\ \nicefrac{1}{2} cup crumbled feta cheese}{Combine first four ingredients in pan. Bring to a boil; cover, reduce heat and simmer 25 minutes or until tender. Drain well. Discard bay leaf. Combine the lentils, cucumber, celery and onion in a medium bowl. Combine orange juice, vinegar and mustard; stir with a whisk. Add to lentil mixture. Stir in cheese. Cover. Chill thoroughly.}

\end{minipage}\par\begin{minipage}{\linewidth}\rtitle{NOMAD SALAD} \index{NOMAD SALAD}
\step{1 \nicefrac{1}{2} cups uncooked bulgur or cracked wheat	
	\\ 1 \nicefrac{1}{2} cups boiling water
	\\ 1 cup dried figs, halved	
	\\ 1 cup chopped fresh parsley
	\\ \nicefrac{1}{2} cup chopped fresh mint	
	\\ \nicefrac{1}{2} cup sweetened dried cranberries
	\\ \nicefrac{1}{2} cup lemon juice	
	\\ 2 Tbsp olive oil
	\\ \nicefrac{1}{2} tsp salt	
	\\ \nicefrac{1}{2} tsp black pepper
	\\ 1 15 \nicefrac{1}{2} oz can chickpeas, drained}{Combine bulgur and boiling water. Cover and let stand 30 minutes. Stir in remaining ingredients. Cover and chill.}

\end{minipage}\par\begin{minipage}{\linewidth}\rtitle{GREEK GREEN BEANS} \index{GREEK GREEN BEANS}
\name{LaDonna Mann's daughter, Barbie, married into a Greek family and learned the ways of Greek cooking.  She made this for us when she was visiting LaDonna in Nairobi. Much to Barbie's surprise, the Kenyan beans took an afternoon of cooking to reach that lovely tender consistency she was aiming for.  The beans were well worth the wait, though, and this recipe has been a family treasure ever since.} \\
\step{1 kilo green beans, cut into 1" lengths
\\ 3 large fresh, juicy tomatoes, cut small
\\ 2 cups onion, chopped
\\ 1 clove garlic, minced
\\ 2 tsp sugar
\\ Salt \& pepper
\\ \nicefrac{3}{4} cup olive oil}{Heat oil. Cook and stir onions and garlic till tender. Add remaining ingredients. Cook until beans are soft. Oregano can be added if desired, although I never do.}

\end{minipage}\par\begin{minipage}{\linewidth}\rtitle{VEGETABLE COUSCOUS} \index{VEGETABLE COUSCOUS}
\step{\nicefrac{1}{2} cup dried chick peas, cooked
	\\ \nicefrac{1}{2} Tbsp olive oil	
	\\ 1 \nicefrac{1}{4} cups coarsely chopped onion
	\\ 2 tsp harissa	
	\\ \nicefrac{1}{8} tsp ground allspice
	\\ salt and pepper	
	\\ 6 carrots, peeled, cut in 2" pieces
	\\ 3 cups 2" cabbage pieces	
	\\ 1 \nicefrac{1}{2} cups cups quartered plum tomatoes	
	\\ 1 \nicefrac{1}{2} cups peeled and cut sweet potatoes
	\\ \nicefrac{1}{2} lb butternut squash, peeled, cut in 1" pieces
	\\ 3 med potatoes, peeled, halved, quartered lengthwise, each piece cut in half.
	\\ 1 \nicefrac{1}{2} cups zucchini pieces	
	\\ 2 cups quick cooking couscous	
	\\ chicken broth or water}{Heat 2 Tbsp oil in very large Dutch oven. Add onion and sprinkle with harissa, allspice, salt and pepper. Saut\'e, stirring until meat has browned, 8-10 minutes. Add 3 cups of water, carrot, cabbage, rutabaga and tomato and bring to a boil. Simmer for 30-45 minutes, until vegetables become tender. Transfer to large bowl and keep warm. Add sweet potato, squash, potato and zucchini to Dutch oven. If necessary, add boiling water to cover vegetables. Bring to boil. Reduce heat to low, partially cover and cook for up to 1 hour or until all vegetables are cooked. Add rest of vegetables and simmer 2--3 minutes. Cook couscous. Stir butter in and heap on large heated serving platter. Moisten couscous with 1 cup of sauce, then arrange vegetables on top. Pour remaining sauce into a heated bowl to pass at the table. Serves 12.}

\end{minipage}\par\begin{minipage}{\linewidth}\rtitle{BUTTERNUT SQUASH COUSCOUS} \index{BUTTERNUT SQUASH COUSCOUS}
\name{The couscous can be replaced with quinoa.}\\
\step{\nicefrac{1}{4} cups sliced almonds
\\ 2 Tbsp olive oil
\\ 2 onions, chopped
\\ 2 cloves garlic, minced
\\ \nicefrac{1}{4} tsp cayenne
\\ \nicefrac{1}{8} tsp grated nutmeg
\\ \nicefrac{1}{8} tsp cinnamon
\\ 1 cup canned diced tomatoes with juice
\\ 1 butternut squash, peeled and cut into \nicefrac{3}{4}" cubes
\\ \nicefrac{1}{4} cup raisins
\\ 3 cups chicken broth
\\ 1 \nicefrac{1}{4} tsp salt
\\ 2 cups drained and rinsed canned chickpeas
\\ \nicefrac{3}{4} cup chopped fresh parsley
\\ 1 \nicefrac{1}{2} cups water
\\ 1 \nicefrac{1}{2} cups couscous}{In a small pan toast the almonds over moderately low heat until golden brown or toast in a 350${}^\circ$F oven. In a dutch oven, heat oil. Add onions and cook until translucent. Add garlic and spices, stirring until fragrant. Stir in tomatoes, squash, raisins, broth and 1 tsp salt. Bring to a simmer. Add chickpeas and cook, covered for 10 minutes. Uncover and continue cooking until squash is tender.  Add parsley.  Meanwhile, bring water with \nicefrac{1}{2} tsp salt to a boil. Stir in the couscous. Remove from heat, cover the pan and let sit for 5 minutes.  Serve couscous with the stew and toasted almonds on top.}

\end{minipage}\par\begin{minipage}{\linewidth}\rtitle{PERSIAN PILAF} \index{PERSIAN PILAF}
\name{(Yvonne Hiebert)}\\
\step{1 small cinnamon stick
\\ 2 tsp cumin seeds
\\ 6 black peppercorns
\\ Seeds of 4 cardamom pods, crushed
\\ 3 cloves
\\ 2 Tbsp oil
\\ 1 small onion, chopped
\\ 1 tsp turmeric
\\ 1 cup rice, uncooked
\\ 2 cups hot veg or chicken stock
\\ 2 bay leaves, torn into pieces
\\ Salt and pepper
\\ Pistachio nuts, coarsely chopped
\\ 2 Tbsp raisins}{Heat a heavy pan. Dry fry cinnamon stick, cumin seeds, peppercorns, cardamom and cloves for 2--3 minutes, until they release their aromas. Add oil to the pan.  Add onion and turmeric. Cook gently for 10 minutes, until onion is softened. Add the rice and stir to coat the grains in oil. Slowly pour in the hot stock, add the bay leaves, salt and pepper and bring to a boil. Lower the heat, cover and cook very gently for about 10 minutes. without lifting the lid. Remove saucepan from heat and leave covered for about 5 minutes. Add the pistachios and raisins to the pilaf.}

\end{minipage}\par\begin{minipage}{\linewidth}\rtitle{LEMON RICE PILAF} \index{LEMON RICE PILAF}
\step{1 \nicefrac{1}{2} tsp olive oil\\
1 clove garlic \\
\nicefrac{1}{2} onion, finely chopped\\
1 \nicefrac{1}{2} cup long grain white rice \\
1 \nicefrac{1}{4} cup chicken broth \\
1 cup water\\
1 large onion \\
1 tsp zest \\
3 Tbsp lemon juice \\
3 Tbsp parsley, finely chopped \\
3 Tbsp dill, finely chopped \\
salt and pepper}{Heat oil over medium heat. Add garlic and onion. Saut\'e 5 minutes until translucent. Add rice. Stir until rice is mostly translucent. Add broth and water. Cover and simmer on low for 12 minutes or until liquid has been absorbed. Remove from heat. Keep covered and rest 10 minutes. Add zest, lemon juice, herbs, salt, and pepper.}

\end{minipage}\par\begin{minipage}{\linewidth}\rtitle{PERSIAN MIRZA GHASEMI} \index{PERSIAN MIRZA GHASEMI}
\name{(A specialty of Masoud Shabani Nezhad, Katrina's Persian friend from grad school.)}\\
\step{2 eggplants
\\ 3--4 garlic cloves
\\ 2 eggs
\\ 1 large tomato
\\ oil
\\ salt \& pepper to taste}{Roast eggplants until completely cooked and blackened. Peel them and chopped until pur\'eed (do NOT blend!). Boil tomato, peel, chop until pur\'eed (do NOT blend). Saut\'e garlic with a little oil. Add eggplant and tomatoes, until water is reduced. Add salt and pepper to taste. Put eggplant mixture in saucepan, break eggs, and then mix to incorporate eggs with eggplant. Cook over medium heat until egg is cooked.}

\end{minipage}\par\begin{minipage}{\linewidth}\rtitle{FALAFELS} \index{FALAFELS}
	\step{540 ml canned chick peas	
	\\ \nicefrac{3}{4} tsp salt
	\\ \nicefrac{3}{4} cup chopped onion	
	\\ \nicefrac{1}{4} tsp pepper
	\\ \nicefrac{3}{4} tsp ground cumin	
	\\ \nicefrac{1}{2} tsp ground coriander
	\\ \nicefrac{1}{2} tsp garlic salt	
	\\ \nicefrac{1}{8} tsp turmeric
	\\ \nicefrac{1}{2} tsp parsley flakes	
	\\ \nicefrac{1}{2} tsp baking powder
	\\ 1.25 cup fine bread crumbs	
	\\ 2 eggs}{Run chick peas and onion through food processor. Add remaining ingredients. Mix well just before frying. Shape into 2-inch patties. Deep fry.}

\end{minipage}\par\begin{minipage}{\linewidth}\rtitle{HUMMUS} \index{HUMMUS}
	\step{4 green onions, cut in 1 \nicefrac{1}{2}" pieces
\\ 2 garlic cloves
\\ \nicefrac{1}{2} tsp ground cumin
\\ \nicefrac{1}{2} tsp salt
\\ 2 15\nicefrac{1}{2}-oz cans chick peas, drained
\\ \nicefrac{1}{4}-\nicefrac{1}{2} cup mayonnaise
\\ \nicefrac{1}{4} cup lemon juice}{Process first 5 ingredients. Add remaining ingredients. Process until smooth.}

\end{minipage}\par\begin{minipage}{\linewidth}\rtitle{BABA GANOUSH} \index{BABA GANOUSH}
\step{2 large eggplants \\
4 medium cloves of garlic, minced\\
6 Tbsp lemon juice \\
\nicefrac{1}{2} cup tahini\\
\nicefrac{3}{4} tsp salt\\
\nicefrac{1}{4} tsp cumin (opt)\\
pinch of smoked paprika (opt)}{Wrap eggplant in foil. Roast at 400$^\circ$F for 45--60 minutes. Scoop out flesh and drain in strainer. Combine all ingredients and mix in food processor or blender.}

\end{minipage}\par\begin{minipage}{\linewidth}\rtitle{TAHINI} \index{TAHINI}
	\step{\nicefrac{1}{2} cup sesame seeds
	\\ 2 Tbsp cooking oi
	\\ \nicefrac{1}{4} cup water}{Spread sesame seeds in ungreased pan. Toast in 350${}^\circ$F oven for 5-10 minutes. Stir seeds every 2 minutes. Place in blender. Add water and oil. Process about 3 minutes. Makes \nicefrac{1}{2} cup.}

\end{minipage}\par\begin{minipage}{\linewidth}\rtitle{GREEK CUCUMBER FETA SAUCE} \index{GREEK CUCUMBER FETA SAUCE}
	\step{1 cup plain yogurt	
	\\ 1 cup sour cream
	\\ 1 tsp white vinegar	
	\\ \nicefrac{1}{2} tsp lemon juice
	\\ 1 small cucumber, peeled, seeded, and finely chopped
	\\ 1 green onion	
	\\ 1 garlic clove, minced
	\\ \nicefrac{1}{4} cup crumbled feta cheese	
	\\ \nicefrac{1}{2} tsp oregano
	\\ \nicefrac{1}{4} tsp lemon zest	
	\\ salt and pepper to taste}{
Mix ingredients and chill 8 hours.}


\end{minipage}\par\begin{minipage}{\linewidth}\rtitle{GREEK POTATOES} \index{GREEK POTATOES}
	\step{8 large potatoes, peeled and cut in large wedges
\\ 4 cloves garlic
\\ \nicefrac{1}{2} cup olive oil
\\ 1 cup water
\\ 1 Tbsp oregano
\\ 1 lemon, juice of
\\ Salt, black pepper}{Combine all ingredients. Bake in pan for 40 minutes. at 420F. Add water if the potatoes get too dry. Stir. Bake another 40 minutes.}

\end{minipage}\par\begin{minipage}{\linewidth}\rtitle{CHICKEN AND OLIVE TAGINE} \index{CHICKEN AND OLIVE TAGINE}
	\step{1.5 chickens rinsed and quartered (3\nicefrac{1}{2} lbs each)
	\\ 2 Tbsp coarsely chopped garlic	
	\\ \nicefrac{1}{2} cup coarsely chopped parsley
	\\ \nicefrac{1}{2} cup coarsely chopped cilantro	
	\\ 1 tsp ground cumin seeds
	\\ 1 tsp olive oil	
	\\ 5 Tbsp lemon juice
	\\ 1 Tbsp pepper	
	\\ 1 \nicefrac{1}{4} cups grated onion
	\\ 2 cups chicken broth	
	\\ 1 preserved lemon, cut into eighths
	\\ 1 cup kalamata olives	
	\\ salt and pepper}{
In food processor, combine the garlic, parsley, cilantro, cumin, olive oil, 1 Tbsp lemon juice and pepper. Pulse or turn on and off for 15 sec. Scrape sides of bowl and process 5 more seconds. Rub resulting paste on all sides of chicken. Place in large glass bowl. Add any remaining paste. Cover and refrigerate for at least 4 hours, preferably overnight.
In large stockpot, place chicken and paste. Add \nicefrac{3}{4} cup grate onion and chicken broth. Cover and simmer 30 minutes. Add remaining onion and simmer 20 minutes more.
Add lemon and olives and simmer 20 minutes. Remove chicken, lemon and olives.
Reduce sauce over med high heat to 1\nicefrac{1}{2} cups, about 5 minutes. Add remaining lemon juice and salt and pepper. Pour sauce over chicken and serve.}

\end{minipage}\par\begin{minipage}{\linewidth}\rtitle{BEEF TAGINE WITH PRUNES} \index{BEEF TAGINE WITH PRUNES}
	\step{16 oz pitted prunes	
	\\ 1 tsp olive oil
	\\ 2 lbs lean beef stew, trimmed and cut into 2" cubes
	\\ 4 cups sliced onion	
	\\ 2 tsp ground coriander
	\\ 1 \nicefrac{1}{2} tsp ground cinnamon	
	\\ 1 tsp ground ginger
	\\ 1 \nicefrac{1}{2} cups beef broth	
	\\ 1 lemon preserved in salt, quartered, pulp discarded
	\\ salt and pepper}{Soak prunes in warm water to cover.
In large casserole, heat oil. Brown meat on all sides. Remove to plate. Add onion and cook until translucent. Add spices and cook 3 minutes. Return meat to casserole. Add beef broth. Bring to a boil. Reduce heat and simmer covered for 45 minutes. Drain prunes and add to stew. Cook for 20 minutes longer or until meat is tender. Add preserved lemon and cook 5 minutes longer. Season with salt and pepper.}


\end{minipage}\par\begin{minipage}{\linewidth}\rtitle{MOROCCAN CHICKEN} \index{MOROCCAN CHICKEN}
	\step{2 Tbsp olive oil	
\\ 8 boneless skinless chicken breasts
\\ 2 onions, chopped	
\\ 1 1-lb butternut squash, peeled, seeded, cut into \nicefrac{3}{4}-inch pieces
\\ \nicefrac{1}{2} tsp cinnamon	
\\ \nicefrac{1}{2} tsp ground cumin
\\ \nicefrac{1}{4} tsp saffron	
\\ 2 cups chicken stock
\\ 4 oz kumquats, quartered lengthwise, seeded (I substituted apricots)
\\ 4 oz pitted prunes	
\\ 2 Tbsp honey
\\ Chopped fresh cilantro}{
Sprinkle chicken with salt and pepper. Brown in oil. Transfer chicken to plate. Pour off all but thin film of fat from skillet. Add onions, and saut\'e on low until very tender and beginning to brown, about 10 minutes. Add squash and stir 2 minutes. Add cinnamon, cumin and saffron and stir until fragrant, about 30 sec. Add stock and bring to boil. Add chicken, kumquats, prunes and honey. Cover and simmer until chicken is cooked through, turning occasionally (about 30 min). If you want the sauce thicker, uncover at the end of the 30 minutes and boil until the right consistency. Season to taste with salt and pepper. Serve on rice or couscous. Sprinkle with cilantro.}

\end{minipage}\par\begin{minipage}{\linewidth}\rtitle{CUMIN-YOGURT CHICKEN} \index{CUMIN-YOGURT CHICKEN}
\step{1 cup sliced onion, separated into rings	
	\\ \nicefrac{1}{2} cup chopped fresh cilantro
	\\ \nicefrac{1}{2} cup plan yoghurt	
	\\ 2 Tbsp ground cumin
	\\ \nicefrac{1}{4} tsp salt	
	\\ \nicefrac{1}{4} tsp ground red pepper
	\\ 2 garlic cloves, crushed	
	\\ 8 chicken thighs
	\\ 2 tsp ground sumac or 1 tsp grated lemon rind}{
Combine first 7 ingredients. Add chicken. Cover and marinate 3 hours. Place chicken and marinade in a baking dish. Sprinkle sumac over chicken mixture. Bake at 350$^\circ$F for 35 minutes. Turn chicken over and bake an additional 35 minutes or until done. Serve over couscous. Can be grilled instead.}

\end{minipage}\par\begin{minipage}{\linewidth}\rtitle{SOUVLAKI} \index{SOUVLAKI}
	\step{6 garlic cloves, sliced in half	
	\\ 1 cup chicken broth
	\\ 1 Tbsp olive oil	
	\\ 1 sprig fresh thyme
	\\ 4 springs fresh rosemary	
	\\ 4 Tbsp fresh lemon juice
	\\ Salt and pepper
	\\ 1 lb lamb shoulder, cut into 1-inch cubes}{
Combine garlic, broth, oil, thyme, rosemary, lemon juice, salt and pepper. Stir well. Add lamb and marinate for several hours or overnight. Divide meat evenly and thread 4 skewers, alternating with pieces of garlic. Wind springs of herbs around the skewers. Reserve the marinade. Grill or broil for 6 to 8 minutes, basting with marinade and turning frequently.}

\end{minipage}\par\begin{minipage}{\linewidth}\rtitle{PITA BREAD} \index{PITA BREAD}
\name{(Yvonne Hiebert)} \\
	\step{2 tsp sugar	
	\\ 2 tsp yeast
	\\ 2 \nicefrac{1}{2} cups water	
	\\ 5-6 cups flour
	\\ 1 Tbsp salt	
	\\ 1 Tbsp oil}{
Mix water with \nicefrac{1}{2} cup flour and yeast. Stir 100 stroke in same direction. Let rest \nicefrac{1}{2} hour. Sprinkle on salt and sugar. Mix in oil. Knead in rest of flour. Rest 1 \nicefrac{1}{2} hours, covered. Can refrigerate up to 7 days at this point. Roll \nicefrac{1}{4}-inch thick ovals. Rest 15 minutes in floured pan. Bake 450$^\circ$F at 5 minutes. Can be baked on BBQ, few minutes on each side.}

\end{minipage}\par\begin{minipage}{\linewidth}\rtitle{PITA BREAD (GF)} \index{PITA BREAD (GF)} \index{Gluten Free!breads}
\name{The secret to getting the pita to puff is having a very hot oven with a preheated baking surface. Wrap baked pitas in a clean towel to keep them soft.}\\
\step{1 \nicefrac{1}{4} cups oat flour
\\ \nicefrac{1}{2} cup potato starch
\\ \nicefrac{1}{3} cup cornstarch
\\ \nicefrac{1}{3} cup tapioca starch
\\ 2 Tbsp psyllium husk powder
\\ 1 Tbsp ground chia seed
\\ 1 Tbsp rapid rise yeast
\\ 1 tsp granulated sugar
\\ 1 tsp salt
\\ 1 cup warm milk
\\ 1 large egg, room temp
\\ \nicefrac{1}{4} cup oil
\\ 1 tsp apple cider vinegar
\\ Rice flour for rolling}{Combine dry ingredients. Whisk together milk, egg, oil and vinegar. Add to flour mixture. Mix on med/low for 5 minutes. This allows the psyllium husk powder and chia seed to absorb the liquid and the dough to thicken. Line 3 baking sheets with parchment paper. Divide the dough into 6 pieces, just a little bigger than a golf ball. Roll the dough into a 6" circle, turning frequently so that it doesn't stick to the counter. Put 2 pitas on each baking sheet.
Lightly cover the pitas with plastic wrap and let sit for 1-1 \nicefrac{1}{2} hours. Place a heavy baking sheet or pizza stone in the oven and preheat the oven to 450${}^\circ$F. Spray two pitas with water, slide out the oven rack and carefully place them on the hot baking sheet. Let the pitas bake 5-7 minutes, or until they are puffed up and lightly browned. Carefully remove the pitas and place them in a clean dry tea towel. Repeat with remaining 4 pitas, letting the oven heat up a few minutes between batches.}


\end{minipage}\par\begin{minipage}{\linewidth}\rtitle{FETA}
(\hyperlink{fetalink}{see Cheeses}) \\
\end{minipage}
