\chapter*{Housekeeping Tips}

\rtitle{LAUNDRY HINTS} \par
\begin{minipage}{\linewidth}\rtitle{Whitening Discoloured Whites}
Combine 1 gal hot water, \nicefrac{1}{2} cup dishwasher soap, \nicefrac{1}{4} cup bleach. Stir well. Soak garment for 30 minutes.  Mom ays, "I soaked my lace on the wedding dress no more than 5 minutes and it was absolutely white."
\end{minipage}\par

\begin{minipage}{\linewidth}\rtitle{Removing fruit stains}
Pour boiling hot water over the stain until it disappears.
\end{minipage}\par

\begin{minipage}{\linewidth}\rtitle{Removing ink stains}
Spray with hairspray and wash out with cold water. Repeat until the stain disappears.
\end{minipage}\par

\begin{minipage}{\linewidth}\rtitle{Removing wax stains}
Place a paper over the stain and iron with a hot iron. Be careful not to scorch the fabric. The paper will lift out the wax. Repeat until nothing more is absorbed into the wax.
\end{minipage}\par

\begin{minipage}{\linewidth}\rtitle{Removing blood stains}
Use cold water, not hot.
\end{minipage}\par

\begin{minipage}{\linewidth}\rtitle{Napkins and Linens}
For stiff cloth napkins, use "spray-on" starch after they are washed and ironed.
Non-iron tablecloths and napkins should be washed in cold water and tumble dried or hung to dry. As soon as they are dry, shake them out and fold them. Do not use hot water or put in dryer at a hot temperature setting or they will crease. Generally a dry-cleaning process is required to remove the creases.
Although white table linens, white bed linens and white towels can seem boring, they are the most practical to keep clean because you can bleach them! Consider white bed linens and white towels for your guests.
\end{minipage}\par

\rtitle{HOUSEHOLD HINTS} \par
\begin{minipage}{\linewidth}\rtitle{Removing rug stains}
Pour a small amount of Rug Doctor concentrate on a cloth. Gently rub the stain until it fades. Take a clean damp cloth and remove the chemical.
\end{minipage}\par

\begin{minipage}{\linewidth}\rtitle{Washing Laminate Floors}
\step{\nicefrac{1}{2} cup vinegar
\\ 1 Tbsp rubbing alcohol
\\ 1 gallon water}{}
\end{minipage}\par

\begin{minipage}{\linewidth}\rtitle{Dish Cloth Care}
If you are using a dish cloth, bleach it regularly.
\end{minipage}\par

\begin{minipage}{\linewidth}\rtitle{Oven and Stove Care}
Line the bottom of your oven with tinfoil to catch the drips. This will make your oven cleaning task much simpler.
Use disposable liners for your stove elements or line the "bowls" with tinfoil.
\end{minipage}\par

\begin{minipage}{\linewidth}\rtitle{China Stains}
Coffee or tea stains in china. Remove with a household cleaner such as comet, Vim or even CLR.
\end{minipage}\par

\rtitle{COOKING TIPS} \par
\begin{minipage}{\linewidth}\rtitle{Icing a cake}
Place the cake on the cake plate. In order to keep the cake plate clean when icing the cake, slide wax paper about \nicefrac{1}{2} inch under the edges of the cake. Ice the cake. Remove the wax paper.
\end{minipage}\par

\begin{minipage}{\linewidth}\rtitle{High altitude}
When cooking cakes, muffins and cookies at high altitudes, add 1 Tbsp flour for every cup in the recipe.
\end{minipage}\par

\begin{minipage}{\linewidth}\rtitle{Types of wheat}
Flour can be made from either hard wheat or soft wheat. In general, flour purchased in Canada is made from hard wheat. Hard wheat dough requires considerably less flour and the texture of the dough should be elastic. Soft wheat dough requires more flour and the dough will generally be sticky rather than elastic. It will take some practice to get the right feel of the dough. If the bread/rolls haven't held their shape when rising, this is generally a sign that you haven't used enough flour. Dough that is too heavy in flour is generally difficult to work with and your hands will get tired rather than exhilarated by the feel of the dough.
\end{minipage}