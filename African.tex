\chapter{African}

\begin{minipage}{\linewidth}\rtitle{PEANUT CHICKEN} \index{PEANUT CHICKEN}
\step{1 chicken, cut up
\\ 1 large onion, minced
\\ 3 large red pimento
\\ 4 tomatoes or \nicefrac{3}{4} cup tomato paste
\\ 1 cup peanut butter mixed with chicken stock
\\ chicken bouillon cubes
\\ salt}{Cover chicken with water and boil. Fry onion, pimento, and tomatoes in oil. Stir in peanut butter, bouillon cubes, and salt. Add chicken. Add more water as needed. Simmer until oil rises to the top of the sauce. \\
OPT: add diced eggplant/garlic/onion greens. \\
OPT: Prepare the sauce and serve over deep-fried sweet potatoes or cassava roots instead of chicken.}

\end{minipage}\par\begin{minipage}{\linewidth}\rtitle{SENEGALESE ONION SAUCE} \index{SENEGALESE ONION SAUCE}
\name{(Hawa Ba)}\\
Hawa's onion sauce is amazing. She wasn't able to give me quantities so the following recipe requires some experimenting.
Slice onions. Marinate with salt, pepper, vinegar and maggi cube.
Saut\'e onions and a very little liquid until onions are brown. Add a little mustard and, if necessary, a little water. 

\end{minipage}\par\begin{minipage}{\linewidth}\rtitle{SUKUMA WIKI} \index{SUKUMA WIKI}
\step{\nicefrac{1}{2} kg sukuma wiki (kale or collard greens)	
	\\ 1--2 chopped onions
	\\ 2--4 chopped tomatoes
	\\ 2--3 Tbsp lard	
	\\ salt and pepper}{
In a large frying pan, fry the onions until soft. Add tomatoes. Cook together until well-heated. Add kale/collard green and cook until soft. Salt to taste. Leftover meat can be added to the dish.}

\end{minipage}\par\begin{minipage}{\linewidth}\rtitle{UGALI} \index{UGALI}
\step{2 cups water	
	\\ \nicefrac{1}{2} tsp salt
	\\ 1 cup white cornmeal}{Bring water and salt to a boil. Stir in cornmeal slowly, sifting it into the water. Reduce heat to medium-low and stir regularly until ugali pulls away from the sides of the pot, about 10 minutes.}

\end{minipage}\par\begin{minipage}{\linewidth}\rtitle{JANE'S SUPU} \index{Supu} \index{Jane} \index{JANE'S SUPU}
Life wouldn't be the same without Jane Nyambura's supu and chapatis---Katrina's favourite.
Shred cabbage, carrots, and onions. Cut up tomatoes very finely. Saut\'e in hot Kimbo (vegetable lard). Add 6-8 cups water (depending on how thick you want the soup). Add cubed potatoes. Mix half a cup water with 1 Tbsp Royco. Simmer until potato is tender. NB: this soup is not the same if you don't have Kenyan Royco on hand.

\end{minipage}\par\begin{minipage}{\linewidth}\rtitle{IRIO} \index{IRIO}
\name{(Chapel Chakula--Anne Wangondu)} \\
\step{1 cup green maize	
	\\ 1 cup peas
	\\ 7 medium potatoes	
	\\ salt to taste}{
Boil maize and peas until soft. In separate pan boil potatoes until cooked thoroughly. Mash maize, peas and potatoes together with salt. Variation: use \nicefrac{1}{2} cup kidney beans instead of peas. Cook chopped pumpkin leaves with potatoes for more colour and flavour.}

\end{minipage}\par\begin{minipage}{\linewidth}\rtitle{SWEET IRIO} \index{SWEET IRIO}
\name{(Beans \& Bananas)}\\
\step{\nicefrac{1}{2} kilo dry beans	\\
5 large green bananas\\
5 ripe bananas}{
Boil beans until tender. Add green bananas. Cook until soft. Add ripe bananas. Cover and boil 10 minutes. Drain excess water. Mash and serve.}

\end{minipage}\par\begin{minipage}{\linewidth}\rtitle{POTATO MASALA} \index{POTATO MASALA} 
\name{(Chapel Chakula)} \\
\step{2 kg potatoes	
	\\ 4 small onions chopped
	\\ 3 tomatoes, chopped	
	\\ 2 Tbsp tomato paste
	\\ \nicefrac{1}{2} cup cream	
	\\ \nicefrac{1}{2} tsp black pepper
	\\ \nicefrac{1}{2} tsp turmeric powder	
	\\ 1 Tbsp Royco Mchuzi mix
	\\ salt to taste}{
Cut the potatoes into cubes and boil until soft but not so soft that they disintegrate. Fry the potatoes, then keep them aside. Fry the onions until golden brown. Add the tomatoes, tomato paste, black pepper, turmeric powder, salt, Royco and cream, then stir to form a masala paste. Add the potatoes to the masala paste, stir until well mixed, cover and cook for about 2 minutes.}

\end{minipage}\par\begin{minipage}{\linewidth}\rtitle{BEEF SAMOSA FILLING} \index{BEEF SAMOSA FILLING}
	\step{500 g ground beef	
		\\ 1 onion
	\\ 2 cloves garlic	
	\\ chilli to taste
	\\ 1 tsp ground turmeric	
	\\ 1 tsp ground coriander
	\\ 2 tsp fresh ginger	
	\\ 50 ml chopped mint
	\\ juice of 1 lemon	
	\\ 2 Tbsp vegetable oil
	\\ salt and pepper to taste}{
Heat the oil in a frying pan, add the onion and garlic. Mix in the spices and seasonings and fry until soft. Add the mince, stirring until cooked. Remove from heat and stir in the mint and lemon juice.}

\end{minipage}\par\begin{minipage}{\linewidth}\rtitle{VEGETABLE SAMOSA FILLING} \index{VEGETABLE SAMOSA FILLING}
	\step{1 potato, finely diced	
		\\ 1 carrot finely diced
	\\ 2 cloves of crushed garlic	
	\\ 1 onion, finely chopped
	\\ 1 cup of frozen peas	
	\\ 1 Tbsp vegetable oil
	\\ 2 tsp curry powder	
	\\ salt and pepper to taste
	\\ 100 ml of vegetable stock}{
Heat oil. Add onion and garlic. Add spices and fry until soft. Add vegetables, seasoning and stir until coated. Add stock. Cover and simmer for 30 minutes until cooked.}

\end{minipage}\par\begin{minipage}{\linewidth}\rtitle{SAMOSA PASTRY} \index{SAMOSA PASTRY}
\name{Kenyan street food.} \\
	\step{1 \nicefrac{1}{2} cup all-purpose flour (Maida)	
		\\ \nicefrac{1}{4} cup oil or lard
	\\ 1 tsp salt
		\\ 6--7 Tbsp water}{
Mix all ingredients. Knead well until a soft, pliable dough. Cover with moist muslin cloth and let rest for 15 minutes.
Divide dough into small balls. Roll each ball into a 4--5-inch circle. Cut the circle in half. Take one semi circle and hold like a cone. Use water to seal. Place a spoon of filling in the cone and seal the third side using a drop of water.
Heat oil and deep fry until golden brown. Serve samosas hot with hari chutney or tamarind chutney.}


\end{minipage}\par\begin{minipage}{\linewidth}\rtitle{NDENGU} \index{NDENGU}
\name{(Chapel Chakula--Zianabu Ikhumba)} \\
\step{1\nicefrac{1}{2} cups green grams	
	\\ \nicefrac{1}{2} onion, chopped finely
	\\ 1 clove garlic, minced	
	\\ 1 tomato, chopped
	\\ \nicefrac{1}{3} cup bell pepper, chopped	
	\\ 1 tsp Royco, mixed with a little water
	\\ 1 chicken or beef cube	
	\\ \nicefrac{1}{2} tsp garam masala, if desired}{
Soak green grams 1-2 hours in hot water to soften before cooking. Cook until very soft. Remove from pan. In same pan, fry onion, garlic, tomato, capsicum and spices for 3-5 minutes. Return grams and liquid to pan and heat through. Good with chapatis or pilau rice.}

\end{minipage}\par\begin{minipage}{\linewidth}\rtitle{MATOKE} \index{MATOKE}
\name{(Chapel Chakula--Liz Oloo)} \\
\step{5-6 green bananas (matoke)	
	\\ \nicefrac{1}{2} onion, chopped
	\\ 1 tomato, chopped	
	\\ 1 tsp Royco
	\\ 1 cup water	
	\\ pinch of chilli powder 
	\\ pinch of turmeric}{
Peel and slice bananas. Fry onion and tomato in a little oil for 2--3 minutes. Add Royco mixed with water. Add chilli powder and turmeric if desired. Add bananas and cook on low heat for 20 minutes or until soft. Add salt to taste.
Variations: spice with curry powder then mash with margarine and milk. Or add 1 cup coconut milk instead of Royco.}

\end{minipage}\par\begin{minipage}{\linewidth}\rtitle{PUMPKIN LEAF IN PEANUT BUTTER} \index{PUMPKIN LEAF IN PEANUT BUTTER}
\name{Substitute spinach for pumpkin leaves if desired.} \\
\step{\nicefrac{1}{2} kilo pumpkin leaves	
	\\ 2 cups water
	\\ \nicefrac{1}{2} tsp baking soda	
	\\ 2 onions, chopped
	\\ 3 mid tomatoes, chopped	
	\\ \nicefrac{1}{2} cup lard or margarine
	\\ salt	
	\\ 1 cup milk
	\\ 1 cup peanut butter}{
Remove rough stems and veins of the leaves. Wash the leaves and cup up finely. Place 2 cups of water in a pan and add \nicefrac{1}{2} tsp soda. Bring to the boil and add the leaves. Cover and cook about 15 minutes. Cook onions and tomatoes in the cooking fat. When they are soft, add the cooked pumpkin leaves and stir together. Add salt, milk and peanut butter. Cook for about 15 minutes on low heat, stirring so the vegetables do not stick to the pan. Serve with ugali or chapatis. NB:  can leave off milk and use water from spinach instead.}

\end{minipage}\par\begin{minipage}{\linewidth}\rtitle{SOMBE (Congolese style)} \index{SOMBE}
\step{cassava
\\ onion, chopped
\\ 2--3 Tbsp palm oil
\\ salt}{
Cook cassava leaves (shredded) and chopped onion in water for at least 1 hour (adding water as leaves boil down). Boil until sombe consistency. Add salt and palm oil. Simmer a few minutes.}

\end{minipage}\par\begin{minipage}{\linewidth}\rtitle{CONGOLESE BEANS} \index{CONGOLESE BEANS}
\step{4-5 cup cooked beans
	\\ 1 onion, chopped
	\\ 75 mL tomato paste
	\\ \nicefrac{1}{4} to \nicefrac{1}{2} cup palm oil}{
Heat oil until it starts to smoke. Add onions. Cook for 1 minutes. Add beans. Cook, stirring for a few minutes. Add some water, tomato paste, and salt. Simmer 5-10 minutes.}

\end{minipage}\par\begin{minipage}{\linewidth}\rtitle{NYANZA LAMB STEW} \index{NYANZA LAMB STEW}
\step{4 Tbsp ghee	\\
600 g boneless shoulder of lamb, cubed\\
salt and pepper	\\
2 onions, sliced\\
3 cloves garlic, chopped	\\
4 tomatoes, chopped and seeded\\
2 tsp ginger, chopped	\\
1 Tbsp finely chopped chillies\\
300 g pumpkin, peeled and cubed	\\
\nicefrac{3}{4} cup water}{In a saucepan, melt 2 Tbsp ghee. Add lamb cubes seasoned with salt and pepper. Brown the lamb. Remove and keep warm. Sauté the onions, garlic, ginger, chillies and tomatoes in remaining ghee for 6-8 minutes. Stir in meat and pumpkin. Cook for 25 minutes, stirring occasionally to keep from burning. Pour on water, cover and cook for 40 minutes, or until the lamb is tender and the pumpkin turns into a puree. Season to taste and serve with irio, ugali or ndizi.}

\end{minipage}\par\begin{minipage}{\linewidth}\rtitle{BABOTIE} \index{BABOTIE}
\name{(Elizabeth Toews)\\
A South African recipe.}\\
\step{2 Tbsp ground ginger
\\ 2 Tbsp brown sugar
\\ 1 Tbsp curry powder
\\ 1 Tbsp turmeric
\\ 2 tsp salt
\\ \nicefrac{1}{2} tsp pepper
\\ 1 Tbsp butter or margarine
\\ 4 med onions, chopped
\\ 2 slices of white bread, soaked in water
\\ 1 kg lean minced beef or lamb
\\ \nicefrac{1}{2} cup raisins
\\ 4 Tbsp chutney
\\ 2 Tbsp smooth apricot jam
\\ 2 Tbsp vinegar
\\ 2 Tbsp Worcestershire sauce
\\ 2 Tbsp tomato paste}{Place all the spices in a warm heavy skillet and heat gently.  Add butter. Stir until melted. Saut\'e onions in mixture.  Squeeze most of water out of bread and add to pan. Add remaining ingredients and cook 20 minutes on low, stirring regularly. Place mixture in baking $9 \times 13$" baking dish.\\
Beat 2 eggs and 1 \nicefrac{1}{2} cups milk together. Pour over the meat mixture. Bake 45 minutes. at 180${}^\circ$C. Serve with rice, sliced bananas and peanuts.}

\end{minipage}\par\begin{minipage}{\linewidth}\rtitle{EAST AFRICAN RICE PILAU} \index{EAST AFRICAN RICE PILAU}
\step{2--3 Tbsp cooking oil/butter/Ghee
\\ 1 tsps cumin spice
\\ \nicefrac{1}{2} tsp cardamom spice
\\ \nicefrac{1}{4} tsp curry spice
\\ \nicefrac{1}{2} tsp star anise
\\ 1 cinnamon stick
\\ 1 bay leaf
\\ 1 tsp pepper flake optional
\\ 1 medium onion chopped
\\ \nicefrac{1}{2} tsp smoked paprika
\\ 2 tsp minced garlic
\\ 1 tsp minced ginger
\\ \nicefrac{1}{2} tsp white/black pepper
\\ \nicefrac{1}{3} -\nicefrac{1}{2} cup cashew optional
\\ 2 cups Basmati rice
\\ 1-2 tomatoes chopped
\\ 2 cups broth
\\ 2 cups coconut milk
\\ 1 \nicefrac{1}{2} tsp salt or more adjust to taste}{Heat a saucepan with oil/ butter. Sauté the cashews for about 2 -3 minutes.
Then add all the spices: cumin, cinnamon stick, bay leaf, curry, cardamom and paprika, stir for about 1 minute. Thourow in the garlic, ginger and onions. Continue cooking for another minute. Add tomatoes and continue cooking for about 2--3 minutes
Stir in rice , cook for about 2 minutes, then add 4 cups of broth /coconut milk, salt and bring to a boil.
Reduce heat, and simmer until rice is completely cooked- about 18- 20 minutes. As the rice cooks you may add more stock if needed. Fluffy, remove cinnamon stick, bay leaf and serve}

\end{minipage}\par\begin{minipage}{\linewidth}\rtitle{CHAPATIS}
(\hyperlink{chapatilink}{see Indian Cuisine}) \\

\end{minipage}