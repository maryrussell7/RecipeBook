\chapter{Ethiopian}

\begin{minipage}{\linewidth}\rtitle{INJERA} \index{INJERA}
\step{1 tsp yeast \\
\nicefrac{1}{2} cup warm water\\
2 cups teff flour \\
1 cup flour \\
3 \nicefrac{1}{2} cups water \\ 
1 tsp salt}{Combine yeast and water in a glass bowl. Cover and set aside for 10 minutes. Mix flours and remaining water to make a runny batter. Stir in activated yeast. Store in loosely covered glass or plastic container in a warm place for 1--3 days. Heat large, ungreased nonstick skillet over medium heat. Add about half a cup of batter. Tilt pan to spread over the bottom of the pan. Cook until all bubbles in batter are popped and bottom of injera is lightly browned. Cover pan with a lid. Remove from heat and let injera steam cook for another minute or two. Remove to platter to cool.}

\end{minipage}\par\begin{minipage}{\linewidth}\rtitle{NITER KEBBEH} \index{NITER KEBBEH}
\name{This can be made ahead and stored in a covered container in the fridge for months.} \\
	\step{1 lb margarine
	\\ 4 Tbsp onion, finely chopped	
	\\ 1 \nicefrac{1}{2} Tbsp garlic, pressed
	\\ 2 tsp fresh ginger, finely grated	
	\\ \nicefrac{1}{2} tsp turmeric
	\\ 4 green cardamom pods, crushed	
	\\ 1 cinnamon stick
	\\ 3 whole cloves	
	\\ \nicefrac{1}{8} tsp ground nutmeg}{
Slowly melt the margarine in a medium sized saucepan over low heat. Add the other ingredients and simmer uncovered on the lowest heat for about 20-30 minutes. Do not let it brown. Strain the mixture through a double layer of cheesecloth, discarding the spices. Refrigerate until set.}

\end{minipage}\par\begin{minipage}{\linewidth}\rtitle{BERBERE} \index{BERBERE}
	\step{1 tsp ground ginger	
	\\ \nicefrac{1}{2} tsp ground cardamom
	\\ \nicefrac{1}{2} tsp ground coriander	
	\\ \nicefrac{1}{2} tsp fenugreek seeds	
	\\ \nicefrac{1}{4} tsp ground nutmeg	
	\\ \nicefrac{1}{8} tsp ground cloves	
	\\ \nicefrac{1}{8} tsp ground cinnamon	
	\\ \nicefrac{1}{8} tsp ground allspice
	\\ 1 Tbsp pressed garlic
	\\ 2 Tbsp finely chopped onions
	\\ 3 Tbsp dry red wine
	\\ 2 Tbsp salt
	\\ 2 cups paprika
	\\ 2 Tbsp ground hot red pepper
	\\ \nicefrac{1}{2} tsp ground black pepper
	\\ 1 \nicefrac{1}{2} cups water	
	\\ 2 Tbsp oil}{
In a heavy 3 qt saucepan, toast the ginger, cardamom, coriander, fenugreek, nutmeg, cloves, cinnamon and allspice over low heat. Remove from heat and let spices cool 5--10 minutes. Blend the toasted spices, onions, garlic, 1 Tbsp of salt and wine. Toast the paprika, red pepper, black pepper and remaining Tbsp of salt over low heat for a minute. Stir in the water, \nicefrac{1}{4} cup at a time, then add the spice-wine mixture. Cook over lowest possible heat for 10--15 minutes. Pack tightly into a jar. When cooled to room temperature, pour oil over to make a film at least \nicefrac{1}{4} inch thick. Refrigerate until ready to use. Replenish the film of oil on top each time you use the berebere. It can be kept in the refrigerator for 5-6 months.}

\end{minipage}\par\begin{minipage}{\linewidth}\rtitle{DORO WAT} \index{DORO WAT}
\step{One 2-\nicefrac{1}{2} lb. chicken, cut into 8 serving pieces
	\\ 2 Tbsp fresh lemon juice	
	\\ 2 tsp salt
	\\ 4 onions, finely chopped	
	\\ \nicefrac{1}{2} cup niter kebbeh
	\\ 6 cloves garlic, minced	
	\\ 2 tsp finely chopped ginger root
	\\ \nicefrac{1}{2} tsp ground fenugreek	
	\\ \nicefrac{1}{2} tsp ground cardamom
	\\ \nicefrac{1}{4} tsp ground nutmeg	
	\\ \nicefrac{1}{2} cup berebere
	\\ 4 Tbsp paprika	
	\\ \nicefrac{1}{2} cup dry red wine
	\\ 1 \nicefrac{1}{2} cup water	
	\\ 4 hard-boiled eggs
	\\ freshly ground black pepper}{
Rinse and dry the chicken pieces. Rub them with lemon juice and salt. Let sit at room temperature for 30 minutes.  
In a heavy enamel stew pot, cook the onions over moderate heat until soft (at least 20 minutes). Do not let brown. Stir in the niter kebbeh. Then add the garlic and spices. Stir well. Add the berebere and paprika, and saut\'e for 3--4 minutes. Pour in the wine and water and bring to a boil. Cook briskly, uncovered, for about 5 minutes. Pat the chicken dry and drop it into the simmering sauce, turning the pieces about until coated on all sides. Reduce the heat, cover, and simmer for 15 minutes. Meanwhile, piece the hard-boiled eggs with the tines of a fork, piercing approximately \nicefrac{1}{4}" into the egg all over the surface. After the chicken has cooked, add the eggs and turn them gently in the sauce. Cover and cook the doro wat for 15 more minutes. Add pepper to taste.}


\end{minipage}\par\begin{minipage}{\linewidth}\rtitle{MINCHET ABISH} \index{MINCHET ABISH}
	\step{1 lb ground beef	
	\\ 1 large red onion, chopped small
	\\ 2 Tbsp berebere sauce	
	\\ \nicefrac{1}{4} cup Niter Kebbeh (spiced butter)	
	\\ 1 clove garlic, crushed
	\\ a few whole cloves	
	\\ \nicefrac{1}{4} tsp grated ginger
	\\ \nicefrac{1}{2} tsp ground cardamom	salt and pepper}{
Brown onions in a little of the Niter Kebbeh but do not fully cook. Add meat and a little more Niter Kebbeh. Brown.  Add remaining ingredients. Fry until brown and dry (at least one hour).}

\end{minipage}\par\begin{minipage}{\linewidth}\rtitle{T'IBS WE'T} \index{T'IBS WE'T}
	\step{1 lb beef, cut into strips	
	\\ 1 med red onion, chopped
	\\ \nicefrac{1}{2} cup niter kebbeh	
	\\ 2 Tbsp berebere
	\\ \nicefrac{1}{4} tsp cardamom	
	\\ \nicefrac{1}{4} tsp ginger
	\\ \nicefrac{1}{8} tsp cumin	
	\\ \nicefrac{1}{8} tsp cloves
	\\ \nicefrac{1}{8} tsp garlic, pressed	
	\\ \nicefrac{1}{8} tsp black pepper
	\\ \nicefrac{1}{4} cup red wine	
	\\ 1 cup water}{
Cook the onions in the niter kebbeh. Add berebere and \nicefrac{1}{4} cup water. Stir. Brown beef in a separate pan. Add the meat to the onions and stir. Add wine, water, spices and salt. Simmer uncovered at low heat for 20 minutes or until sauce has the right consistency.}

\end{minipage}\par\begin{minipage}{\linewidth}\rtitle{AYIB BEGOMEN} \index{AYIB BEGOMEN}
	\step{200 grams cottage cheese	
	\\ 1 Tbsp Niter Kebbeh
	\\ 1 clove garlic, pressed	
	\\ \nicefrac{1}{8} tsp ground cardamom
	\\ salt \& pepper to taste	
	\\ 1 bunch collard greens, chopped
	\\ 1 medium onion, chopped	
	\\ 1 Tbsp niter kebbeh}{
Mix cottage cheese, first amount of niter kebbeh, garlic, cardamom and salt and pepper. Set aside.
Wash greens. Chop coarsely. Heat niter kebbeh in saucepan. Add onion, chopped greens and some salt. Saut\'e until onion turns clear and stems have softened. Cook with lid until to allow greens to steam. Add a small amount of water if necessary. Just before serving, add the cottage cheese mixture.}

\end{minipage}\par\begin{minipage}{\linewidth}\rtitle{MISR WAT} \index{MISR WAT}
\step{4 Tbsp olive oil
\\ 2 med onions, chopped
\\ 5 cloves garlic, minced
\\ 1 cup red lentils
\\ \nicefrac{1}{2} tsp ground cardamom
\\ 3 Tbsp berebere
\\ 1 tsp salt or to taste}{Heat a large sauce pan (make sure it isn't one with a nonstick surface) over med to med-high heat. Cook the onions without oil until translucent. Add oil and saut\'e a few more minutes, stirring frequently. Add garlic, saut\'e a few additional minutes. Add cardamom, berbere, and a couple teaspoons water. Reduce heat and stir frequently for about 10 minutes, adding water as necessary. Add \nicefrac{1}{4} cup water and lentils. Saut\'e, stirring almost constantly, until water is absorbed and lentils are starting to stick to the bottom. Continue adding water \nicefrac{1}{4} cup at a time. Scrape the caramelized lentil gunk into the rest of the wot to give it a sweet, nutty flavor. Continue doing this until the lentils are fully cooked. Top off with enough water to give it a nice wot consistency. Adjust seasoning and add salt.}

\end{minipage}\par\begin{minipage}{\linewidth}\rtitle{ETHIOPIAN VEGETABLES} \index{ETHIOPIAN VEGETABLES}
\step{potatoes, peeled and cut in chunks \\
carrots, peeled and cut lengthwise\\
cabbage\\
green beans \\
niter kibbeh\\	
salt and pepper to taste}{Boil vegetables. Drain and add niter kibbeh.}

\end{minipage}\par\begin{minipage}{\linewidth}\rtitle{SHIRO} \index{SHIRO}
	\step{\nicefrac{1}{2} to 1 onion, finely minced	
	\\ 2 tsp olive oil
	\\ \nicefrac{1}{4} to \nicefrac{1}{2} tsp minced garlic	
	\\ 2 cups water
	\\ \nicefrac{1}{2} cup shiro powder}{Soften onion in oil. Add garlic. Add water. Bring to boil.  Sprinkle shiro into water, while whisking continually, so that it doesn't clump.}

\end{minipage}\par\begin{minipage}{\linewidth}\rtitle{SALATA} \index{SALATA}
	\step{iceberg lettuce \\
tomatoes \\
jalapenos \\
olive oil \\
lemon juice \\
ginger (opt) \\
garlic \\
salt}{Make salad from lettuce, tomatoes, and jalapenos. Mix oil, lemon juice, ginger, garlic, and salt to taste for dressing.}

\end{minipage}