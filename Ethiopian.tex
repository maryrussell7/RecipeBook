\chapter{Ethiopian}

\begin{minipage}{\linewidth}\rtitle{INJERA} \index{INJERA} \index{Gluten Free}
\step{1 \nicefrac{1}{2} cups Teff flour
\\ 2 cups water
\\ \nicefrac{1}{2} tsp baking powder
\\ \nicefrac{1}{4} tsp salt}{Combine teff and water in a glass bowl. Cover and let sit for 24 hrs. Stir in baking powder and salt. Heat a lightly greased pan over med/low heat. Pour in batter to cover the bottom of the pan. Cover with a lid. Bake until top is bubbled and dry.  Do not flip the injera.}

\end{minipage}\par\begin{minipage}{\linewidth}\rtitle{NITER KEBBEH} \index{NITER KEBBEH}
\textit{This can be made ahead and stored in a covered container in the fridge for months.} \\
	\step{1 lb margarine
	\\ 4 Tbsp onion, finely chopped	
	\\ 1 \nicefrac{1}{2} Tbsp garlic, pressed
	\\ 2 tsp fresh ginger, finely grated	
	\\ \nicefrac{1}{2} tsp turmeric
	\\ 4 green cardamom pods, crushed	
	\\ 1 cinnamon stick
	\\ 3 whole cloves	
	\\ \nicefrac{1}{8} tsp ground nutmeg}{
Slowly melt the margarine in a medium sized saucepan over low heat. Add the other ingredients and simmer uncovered on the lowest heat for about 20-30 min. Do not let it brown. Strain the mixture through a double layer of cheesecloth, discarding the spices. Refrigerate until set.}

\end{minipage}\par\begin{minipage}{\linewidth}\rtitle{BERBERE} \index{BERBERE}
	\step{1 tsp ground ginger	
	\\ 1 Tbsp pressed garlic
	\\ \nicefrac{1}{2} tsp ground cardamom	
	\\ 2 Tbsp salt
	\\ \nicefrac{1}{2} tsp ground coriander	
	\\ 3 Tbsp dry red wine
	\\ \nicefrac{1}{2} tsp fenugreek seeds	
	\\ 2 cups paprika
	\\ \nicefrac{1}{4} tsp ground nutmeg	
	\\ 2 Tbsp ground hot red pepper
	\\ \nicefrac{1}{8} tsp ground cloves	
	\\ \nicefrac{1}{2} tsp ground black pepper
	\\ \nicefrac{1}{8} tsp ground cinnamon	
	\\ \nicefrac{1}{8} tsp ground allspice
	\\ 1 \nicefrac{1}{2} cups water	
	\\ 2 Tbsp finely chopped onions
	\\ 2 Tbsp oil}{
In a heavy 3 qt saucepan, toast the ginger, cardamom, coriander, fenugreek, nutmeg, cloves, cinnamon and allspice over low heat for a minute or so, stirring constantly until heated through. Remove pan from heat and let spices cool 5-10 min.
Combine the toasted spices, onions, garlic, 1 Tbsp of salt and wine. Blend in electric blender to make a paste.
Combine the paprika, red pepper, black pepper and remaining Tbsp of salt in saucepan and toast over low heat for a minute, until heated through, shaking the pan and stirring the spices constantly.
Stir in the water, \nicefrac{1}{4} cup at a time, then add the spice-wine mixture. Cook over lowest possible heat for 10-15 min.
Pack tightly into a jar or crock. When cooled to room temperature, pour oil over to make a film at least \nicefrac{1}{4} inch thick. Cover and refrigerate until ready to use. If you replenish the film of oil on top each time you use the berebere, it can be kept in the refrigerator for 5-6 months.}

\end{minipage}\par\begin{minipage}{\linewidth}\rtitle{DORO WAT} \index{DORO WAT}
\step{One 2-\nicefrac{1}{2} lb. chicken, cut into 8 serving pieces
	\\ 4 Tbsp fresh lemon juice	
	\\ 4 tsp salt
	\\ 4 onions, finely chopped	
	\\ \nicefrac{1}{2} cup niter kebbeh
	\\ 6 cloves garlic, minced	
	\\ 2 tsp finely chopped ginger root
	\\ \nicefrac{1}{2} tsp ground fenugreek	
	\\ \nicefrac{1}{2} tsp ground cardamom
	\\ \nicefrac{1}{4} tsp ground nutmeg	
	\\ \nicefrac{1}{2} cup berebere
	\\ 4 Tbsp paprika	
	\\ \nicefrac{1}{2} cup dry red wine
	\\ 1 \nicefrac{1}{2} cup water	
	\\ 4 hard-boiled eggs
	\\ freshly ground black pepper}{
Rinse and dry the chicken pieces. Rub them with lemon juice and salt. Let sit at room temperature for 30 min.  
In a heavy enamel stew pot, cook the onions over moderate heat for about 5 min. Do not let brown or burn. Stir in the niter kebbeh. Then add the garlic and spices. Stir well. Add the berebere and paprika, and saut\'e for 3-4 min. Pour in the wine and water and bring to a boil. Cook briskly, uncovered, for about 5 min. Pat the chicken dry and drop it into the simmering sauce, turning the pieces about until coated on all sides. Reduce the heat, cover, and simmer for 15 min. Meanwhile, piece the hard-boiled eggs with the tines of a fork, piercing approximately \nicefrac{1}{4}" into the egg all over the surface. After the chicken has cooked, add the eggs and turn them gently in the sauce. Cover and cook the doro wat for 15 more min. Add pepper to taste.}


\end{minipage}\par\begin{minipage}{\linewidth}\rtitle{MINCHET ABISH} \index{MINCHET ABISH}
	\step{1 lb ground beef	
	\\ 1 large red onion, chopped small
	\\ 2 Tbsp berebere sauce	
	\\ \nicefrac{1}{4} cup Niter Kebbeh (spiced butter)	
	\\ 1 clove garlic, crushed
	\\ a few whole cloves	
	\\ \nicefrac{1}{4} tsp grated ginger
	\\ \nicefrac{1}{2} tsp ground cardamom	salt and pepper}{
Brown onions in a little of the Niter Kebbeh but do not fully cook. Add meat and a little more Niter Kebbeh. Brown.  Add remaining ingredients. Fry until brown and dry (at least one hour).}

\end{minipage}\par\begin{minipage}{\linewidth}\rtitle{T'IBS WE'T} \index{T'IBS WE'T}
	\step{1 lb beef, cut into strips	
	\\ 1 med red onion, chopped
	\\ 1 cup niter kebbeh	
	\\ 2 Tbsp berebere
	\\ \nicefrac{1}{4} tsp cardamom	
	\\ \nicefrac{1}{4} tsp ginger
	\\ \nicefrac{1}{8} tsp cumin	
	\\ \nicefrac{1}{8} tsp cloves
	\\ \nicefrac{1}{8} tsp garlic, pressed	
	\\ \nicefrac{1}{8} tsp black pepper
	\\ \nicefrac{1}{4} cup red wine	
	\\ 1 cup water}{
Cook the onions in the niter kebbeh. Add berebere and \nicefrac{1}{4} cup water. Stir. Brown beef in a separate pan. Add the meat to the onions and stir. Add wine, water, spices and salt. Simmer uncovered at low heat for 20 min or until sauce has the right consistency. The sauce will reduce as it cooks. (Note: although the recipe calls for 1 cup of niter kebbeh, I suggest reducing that to \nicefrac{1}{2} cup.)}

\end{minipage}\par\begin{minipage}{\linewidth}\rtitle{YEGOMEN KITFO} \index{YEGOMEN KITFO}
	\step{200 grams cottage cheese	
	\\ 1 Tbsp Niter Kebbeh
	\\ 1 clove garlic, pressed	
	\\ \nicefrac{1}{8} tsp ground cardamom
	\\ salt \& pepper to taste	
	\\ 1 bunch collard greens, chopped
	\\ 1 medium onion, chopped	
	\\ 1 Tbsp niter kebbeh}{
Mix cottage cheese, first amount of niter kebbeh, garlic, cardamom and salt and pepper. Set aside.
Wash greens. Chop coarsely. Heat niter kebbeh in saucepan. Add onion, chopped greens and some salt. Saut\'e until onion turns clear and stems have softened. Cook with lid until to allow greens to steam. Add a small amount of water if necessary. Just before serving, add the cottage cheese mixture.}

\end{minipage}\par\begin{minipage}{\linewidth}\rtitle{MISR WAT} \index{MISR WAT}
\step{4 Tbsp olive oil
\\ 2 med onions, chopped
\\ 5 cloves garlic, minced
\\ 1 cup red lentils
\\ \nicefrac{1}{2} tsp ground cardamom
\\ 3 tbsp berebere
\\ 1 tsp salt or to taste}{Heat a large sauce pan (make sure it isn't one with a nonstick surface) over med to med-high heat. Cook the onions without oil.  Stir constantly until they have become translucent. Add oil and saute a few more minutes, stirring frequently. Add garlic, saute a few additional minutes. Add the cardamom and berebere as well as a couple tsps water. Reduce your heat a notch and stir frequently for about 10 minutes, adding an additional splash of water if necessary to avoid sticking. Add \nicefrac{1}{4} cup water and lentils. Saute, stirring almost constantly, until water is absorbed and lentils are starting to stick to the bottom. Continue adding water about \nicefrac{1}{4} cup at a time. Scrape the caramelized lentil gunk into the rest of the wot to give it a sweet, nutty flavor. Continue doing this until the lentils are full cooked. Top off with enough water to give it a nice wot consistency. Adjust seasoning and add salt.  (opt:  add a bit of sugar, if necessary)}

\end{minipage}\par\begin{minipage}{\linewidth}\rtitle{ETHIOPIAN VEGETABLES} \index{ETHIOPIAN VEGETABLES}
\step{potatoes, peeled and cut in chunks \\
carrots, peeled and cut lengthwise\\
cabbage\\
green beans (trim the ends)\\
beets\\
niter kibbeh\\	
salt and pepper to taste}{Boil vegetables. Drain and add niter kibbeh.}

\end{minipage}\par\begin{minipage}{\linewidth}\rtitle{ETHIOPIAN VEGETABLE STEW} \index{ETHIOPIAN VEGETABLE STEW}
\textit{This is not an original Ethiopian recipe as far as I can tell, but it sure tastes good and it was called Ethiopian Vegetable Bowl.} \\
	\step{\nicefrac{1}{4} cup vegetable oil 	
	\\ \nicefrac{1}{2} tsp ground ginger
\\ \nicefrac{1}{2} tsp ground turmeric	
\\ \nicefrac{1}{2} tsp ground black pepper
\\ 1 tsp ground cloves	
\\ 1 tsp fenugreek seeds
\\ 1 head garlic, minced	
\\ 1 tsp salt
\\ 3 large onions, chopped	
\\ 4 large carrots, cubed
\\ 4 large potatoes, cubed	
\\ \nicefrac{1}{4} head cabbage, chopped
\\ 2 cups tomato puree	
\\ 2 cups water
\\ salt and pepper to taste	}{
Heat the oil in a large skillet over medium-high heat. Stir in the ginger, turmeric, black pepper, cloves, fenugreek, garlic, and one tsp salt. Continue to stir until the spices and garlic are well coated in oil, about 30 seconds. Stir in the onions; cook, stirring, until translucent, about 5 min. Add the carrots, potatoes, and cabbage; cook, stirring frequently, until the vegetables begin to soften, about 3 min. 
Stir in the tomato puree and the water. Continue to cook over very low heat, until vegetables are soft and the tomato sauce thickens, about 30 to 40 min. Taste for seasoning and add additional salt and pepper, if needed. }

\end{minipage}\par\begin{minipage}{\linewidth}\rtitle{ETHIOPIAN CABBAGE DISH} \index{ETHIOPIAN CABBAGE DISH}
\step{\nicefrac{1}{2} cup olive oil
\\ 4 carrots, thinly sliced
\\ 1 onion, thinly sliced
\\ 1 tsp sea salt
\\ \nicefrac{1}{2} tsp ground black pepper
\\ \nicefrac{1}{2} tsp ground cumin
\\ \nicefrac{1}{4} tsp ground turmeric
\\ \nicefrac{1}{2} head cabbage shredded
\\ 5 potatoes, peeled and cut into 1" cubes}{Heat the olive oil in a skillet over med heat. Cook the carrots and onion in the hot oil about 5 minutes. Stir in the spices and cabbage. Cook 15-20 min. Add the potatoes. Cover and reduce heat to med-low. Cook until the potatoes are soft (about 20-30 minutes)}

\end{minipage}\par\begin{minipage}{\linewidth}\rtitle{SHIRO} \index{SHIRO}
	\step{\nicefrac{1}{2} to 1 onion, finely minced	
		\\ \nicefrac{1}{4} to \nicefrac{1}{2} tsp minced garlic
	\\ 2 tsp olive oil	
	\\ \nicefrac{1}{2}--1 cup shiro powder}{Soften onion in oil. Add garlic. Add water. Bring to boil.  Sprinkle shiro into water, while whisking continually, so that it doesn't clump. Should be consistency of creamy hummus.  Add 1 Tbsp of berebere spice if shiro powder mix is unspiced. If shiro powder is too spicy, replace shiro powder with chickpea flour.}

\end{minipage}\par\begin{minipage}{\linewidth}\rtitle{SALATA} \index{SALATA}
	\step{lettuce \\
tomatoes\\
jalapenos\\
olive oil\\
lemon juice\\
ginger\\
garlic\\
salt}{Make salad from lettuce, tomatoes, and jalapenos. Mix oil, lemon juice, ginger, garlic, and salt to taste for dressing.}

\end{minipage}